\dish{Pain au raisin de Cyril Lignac}
\altdish{Cyril Lignac, pain au raisin}
%\serves{}
\fait{10--12}
\prep{8 heures}
\source{cyril-lignac-patisserie}

% https://www.youtube.com/watch?v=qhWU2nHk8Mg

\stage{Détrempe}
\begin{ingredients}
  \ingr{140}{ml}{eau}
  \ingr{10}{g}{levure sèche}
  \ingr{100}{g}{lait, à température ambiante}
  \ingr{500}{g}{farine T45}
  \ingr{12}{g}{sel}
  \ingrS{60}{g}{sucre}
  \ingrS{35}{g}{beurre doux, frais, coupé en morceaux}
  \ingr{240}{g}{beurre doux, frais}
\end{ingredients}


\begin{recipe}
  \begin{enumerate}

  \item Verser la levure sur l'eau.  Ajouter le lait, farine, sel et
    sucre.  Vitesse 1 pendant 4 minutes.  Ajouter les morceaux de
    beurre petit à petit.  Pétrir 10~minutes à vitesse moyenne.

  \item Filmer, laisser reposer 30 minutes à température ambiante.

  \item Dégazer, rouler à $30 \times 20$ cm.  Filmer, réserver 2 heures au frais.

  \item Beurre $20 \times 20$ cm en papier cuisson.

  \item Étaler détrempe à $40 \times 20$ cm.  Poser carré de beurre au
    centre, rabattre et fermer.

  \item Étaler à $60 \times 20$ cm, faire un tour double.  Filmer et
    laisser 30~minutes au frais.

  \item Rouler à $60 \times 20$ cm, faire un tour simple.  Laisser
    reposer 30~minutes au frais.
  \end{enumerate}
\end{recipe}

\stage{Crème d'amande}
\begin{ingredients}
  \ingr{125}{g}{beurre pommade}
  \ingr{125}{g}{sucre glace}
  \ingr{15}{g}{maïzena}
  \ingr{160}{g}{amande en poudre}
  \ingr{90}{g}{oeufs à température ambiante (2 oeufs)}
\end{ingredients}

\begin{recipe}
  \begin{enumerate}
  \item Mixer le beurre pommade, sucre glace, maïzena et poudre
    d'amande sans foisonner.  Ajouter les oeufs un à un.

  \item Filmer et mettre au frais.
  \end{enumerate}
\end{recipe}

\stage{Crème pâtissière}
\begin{ingredients}
  \ingr{150}{g}{lait}
  \ingrS{}{}{vanille, demi-gousse}
  \ingr{25}{g}{jaune d'oeuf (jaune d'1 oeuf)}
  \ingr{30}{g}{sucre}
  \ingr{10}{g}{farine, T55}
  \ingr{5}{g}{maïzena}
  \ingrS{}{}{poudre à crème}
  \ingr{15}{g}{beurre doux en morceaux}
\end{ingredients}

\begin{recipe}
  \begin{enumerate}
  \item Mélanger le lait et la vanille.  Chauffer et laisser infuser
    30~minutes.

  \item Mélanger le jaune d'oeuf, sucre, farine et poudre à crème.  Chinoiser, puis
    ajouter le lait à la vanille.

  \item Remettre sur feu, remuer 3 minutes.

  \item Hors four, ajouter le beurre.  Filmer.
  \end{enumerate}
\end{recipe}

\stage{Sirop}
\begin{ingredients}
  \ingr{100}{g}{sucre}
  \ingr{100}{ml}{eau}
\end{ingredients}

\stage{Assemblage}
\begin{ingredients}
  \ingrS{120}{g}{raisins secs}
  \ingr{1}{}{oeuf}
  \ingr{1}{}{jaune d'oeuf}
\end{ingredients}

\begin{recipe}
  \begin{enumerate}
  \item Étaler la détrempe à 3~mm d'épaisseur.

  \item Couper les bords pour montrer la feuilletage.

  \item Mélanger la crème pâtissière et la crème d'amande pour les
    détendre, plus mélanger-les ensemble.

  \item Aplatisser le haut de la détrempe de 2 cm.  Mouiller avec un
    pinceau.  Étaler la crème, puis les raisins secs.  Rouler puis
    filmer et laisser au moins 1~heure au frigo.

  \item Détailler le rouleau à 2,5~cm d'épaisseur.

  \item Dorer avec l'oeuf, puis laisser reposer 30~minutes à 30\degreeC{}.

  \item Préchauffer le four à 175\degreeC{}.

  \item Enfourner 20--25~minutes.

  \item À la sortie du four, avec l'aide d'un pinceau, lustrer avec le sirop.

  \end{enumerate}
\end{recipe}

%\accord{}

