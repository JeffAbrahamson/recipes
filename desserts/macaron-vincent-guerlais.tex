\dish{Macarons de Vincent Guerlais}
\altdish{Vincent Guerlais, Macarons}
%\serves{}
%\makes{}
\prep{1h}
\source{vincent.guerlais}


\stage{Macarons Natures}

\begin{ingredients}
  \ingr{205}{g}{sucre glace}
  \ingr{205}{g}{poudre d'amande}
  \ingr{\it 25}{\it g}{\textit{(cacao, si macarons au chocolat)}}\\[-2mm]
  \ingr{160}{g}{sucre semoule}
  \ingr{37}{g}{eau}\\[-2mm]
  \ingr{72}{g}{blancs d'oeufs, mont\'es}
  \ingr{60}{g}{blancs d'oeufs, crus \textit{(si chocolat, 65~g)}}
  \ingr{qs}{}{vanille}
\end{ingredients}

{\it Pour faire des macarons au chocolat, passer \`a 65~g blancs
  d'ouefs crus et ajouter 25~g de cacao poudre dans la poudre
  d'amandes.}

\begin{recipe}
  \begin{enumerate}

  \item Passer le sucre glace et la poudre d'amende dans un robo pour
    faire un poudre tr\`es fin.  Les tamiser.

  \item Cuire le sucre semoule et l'eau \`a 118\C, puis verser sur les
    blancs mont\'es en vitesse moyenne.

  \item Ajouter les blancs crus \`a la meringue, puis m\'elanger les
    poudres et la vanille et macaroner.

  \item Dresser sur papier cuisson.

  \item Laisser s\'echer au moins 20~minutes, puis cuire
    10--11~minutes \`a 160\C\ (ventil\'e) ou 180\C\ (statique).
  \end{enumerate}
\end{recipe}


\stage{Garniture Fraise Menthe}

\begin{ingredients}
  \ingr{225}{g}{fraises (m\^eme congel\'ees), pass\'ees pr\'ealablement au blender}
  \ingr{60}{g}{sucre}
  \ingr{35}{g}{glucose (DE 40)}
  \ingr{4}{g}{pectine NH}
  \ingr{20}{g}{jus de citron}
  \ingr{55}{g}{chocolat blanc (beurre de cacao?)}
  \ingr{10}{g}{Get 27}
\end{ingredients}


\begin{recipe}
  \begin{enumerate}

  \item Faire cuire les fraises avec le sucre et le glucose.  Vers
    60\C\ (temp\'erature pr\'ecise pas tr\`es importante) ajouter la
    pectine.  Cuire jusqu'\`a 103\C.

  \item Ajouter le jus de citron, puis verser d\'elicatement sur le
    chocolat blancs (beurre de cacao~?).  Ajouter le Get~27.

  \item Laisser refroidir avant de dresser les coques de macarons.
  \end{enumerate}  
\end{recipe}


\stage{Garniture Chocolat}

\begin{ingredients}
  \ingr{125}{g}{cr\`eme fleurette}
  \ingr{100}{g}{chocolat pur Venezuela 72\%}
  \ingr{15}{g}{beurre}
\end{ingredients}


\begin{recipe}
  \begin{enumerate}

  \item Faire une ganache.  Laisse  refroidir.

  \end{enumerate}
\end{recipe}
