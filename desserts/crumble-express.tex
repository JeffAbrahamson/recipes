\dish{Crumble de Pommes Express}
\altdish{Express Crumble de Pommes}
\altdish{Pommes, Crumble Express}
\altdish{Apple Crumble Express}
\pour{4}
\prep{20 minutes + 40 minutes cuisson}
\source{CVF-99}

\begin{ingredients}
  \ingr{6}{}{pommes acidulées}
  \ingr{100}{g}{sucre en poudre}
  \ingr{100}{g}{farine}
  \ingr{100}{g}{beurre salé}
  \ingr{}{}{raisins secs}
  \ingr{}{}{jus de citron}
  \ingr{}{}{cannelle}
  \ingr{}{}{muscade}
\end{ingredients}


\begin{recipe}
  \begin{enumerate}

  \item Préchauffer le four à 180\degreeC.

  \item Peler les pommes et éliminer le coeur à l'aide d'un
    vide-pommes.  Les couper en deux.

  \item Mettre 100~g de sucre, la farine, et le beurre coupé en
    morceaux dans un mixeur équipé d'un couteau.  Faire
    fonctionner l'appareil jusqu'à obtenir un sable.

  \item Poser les pommes côté bombé vers le haut dans un moule à manqué.

  \item Répartisser le sable sur les pommes et enfourner pour 40
    minutes environ, jusqu'à ce que le dessus soit doré.

  \item Servir chaud ou tiède, nature ou accompané d'une boule de
    glace à la vanille ou de la crème fraîche.

  \end{enumerate}
\end{recipe}

Notes:
\begin{itemize}
\item De bonnes pommes à choisir : la reine des reinettes, la
  golden, la gala, la granny (ajouter plus de sucre)
\item De mauvaises pommes à choisir : la rouge américaine
\item La recette originale propose en plus un caramel de 100~g de
  sucre et une \ccf\ d'eau.  Je trouve que c'est mieux sans.  Elle ne
  proposait que 75~g de beurre.  Elle ne propose pas de raisin sec,
  jus de citron, cannelle, ni muscade.
\end{itemize}
