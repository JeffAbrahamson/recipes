\dish{Rugelach}
\altdish{ראגעלעך}               % Yiddsh.
\altdish{רוגלך}                 % Hebrew.
%\serves{}
\makes{24}
\prep{1 heure sur 4 heures}
\source{jewish-bakery}

\textbf{pâte}
\begin{ingredients}
  \ingr{4}{g}{levure sèche}
  \ingrS{100}{ml}{lait entier}
  \ingr{250}{g}{farine T45}
  \ingr{1}{}{oeuf}
  \ingr{50}{g}{sucre}
  \ingrS{5}{g}{sel}
  \ingrS{90}{g}{beurre}
  \ingr{1}{}{oeuf pour la dorure}
\end{ingredients}

\textbf{garniture}
\begin{ingredients}
  \ingr{50}{g}{beurre mou}
  \ingr{50}{g}{sucre}
  \ingr{25}{g}{cacao en poudre}
\end{ingredients}

\textbf{imbibage}
\begin{ingredients}
  \ingr{100}{ml}{eau}
  \ingr{20}{g}{sucre}
\end{ingredients}

\begin{recipe}
  \begin{enumerate}

  \item Délayer la levure dans le lait, ajouter une pincée de farine,
    et laisser mousser 15~minutes.

  \item Ajouter farine, oeuf, sucre, et sel.  Pétrir 2~minutes en
    basse vitesse, puis ajouter le beurre et pétrir 10~minutes en
    vitesse moyenne.

  \item Faire une boule.  Filmer et laisser 2~heures à 20\degreeC{} ou
    toute la nuit au réfrigérateur.

  \item Pour la garniture, mélanger le beurre, le sucre et le cacao
    jusqu'à obtenir une pâte homogène

  \item Étaler la pâte en un rectangle de 5~mm d'épaisseur en
    proportion $2\times 1$.

  \item Étaler la garniture sur la partie supérieure, puis rabattre la
    partie inférieure sur la moitié garnie.  Tourner le pâton un quart
    de tour.

  \item Étaler la pâte encore à 5~mm, plier en trois, les deux
    extrémités vers le milieu.  C'est un tour simple.

  \item Faire un deuxième tour simple, filmer, et mettre au frigo
    30~minutes.

  \item Étaler encore à 5~mm, couper en deux pour obtenir deux bandes
    de 12~cm de large $\times$ 30~cm de longue.

  \item Détailler des triangles de 5 cm $\times$ 12~cm.  Rouler chaque
    triangle comme des croissants en tirant très légèrement sur le
    triangle pour l'agrandir : côté 5~cm vers la pointe.  Déposer sur
    une plaque recouverte de papier cuisson.

  \item Dorer avec l'oeuf, laisser reposer une heure à température
    ambiant.

  \item Préchauffer le four à 165\degreeC.

  \item Dorer les rugelach une deuxième fois, puis enfourner pour
    25~minutes.

  \item Dans une casserole porter l'eau et le sucre à l'ébullition,
    puis laisser refroidir.

  \item À la sortie du four, napper les rugelach de sirop.

  \item Les rugelach se dégustent chaud ou froid.


  \end{enumerate}
\end{recipe}

\textit{La recette originelle proposer un sirop à 250~ml d'eau et 50~ml de sucre.}

%\accord{}
