\dish{Mousse au chocolat noir}
\altdish{Mousse au chocolat, Valrhona}
\altdish{Valrhona, mousse au chocolat}
\serves{8 personnes}
%\makes{}
\prep{15 minutes}
\source{valrhona-2016}

\stage{Mousse au chocolat}

\begin{ingredients}
  \ingr{300}{g}{chocolat guanaja 70\%}
  \ingr{150}{g}{crème liquide entière}
  \ingr{60}{g}{jaune d'œufs}
  \ingr{200}{g}{blanc d'œufs}
  \ingr{50}{g}{sucre semoule}
\end{ingredients}


\begin{recipe}
  \begin{enumerate}

  \item Faire fondre le chocolat, ajouter la crème bouillante.  En
    dessous de 60 \degreeC, ajouter les jaunes d'œufs.

  \item Parallèlement, monter les blancs en neige en ajoutant une
    toute petite partie du sucre dès le départ et le reste sur la fin.

  \item Incorporer les blancs dans le chocolat.

  \item Mouler et mettre au réfrigérateur pendant 12 heures.

  \end{enumerate}
\end{recipe}

\stage{Sauce chocolat}

\begin{ingredients}
  \ingr{85}{g}{chocolat guanaja \%70}
  \ingr{100}{g}{lait entier}
\end{ingredients}

\begin{recipe}
  \begin{enumerate}

  \item Faire fondre le chocolat, incorporer le lait bouillant.
    Réserver au réfrigérateur ou servir bien chaud.

  \end{enumerate}
\end{recipe}

\stage{Service}

Trente minutes avant la dégustation, sortir les mousses du
réfrigérateur et accompagner de la sauce froide ou chaude.

À noter : la présence du jaune d'œuf cru, cette mousse ne conserve
pas au-délà de 24 heures.

%\accord{}
