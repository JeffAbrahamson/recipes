\dish{Nib buns}
\altdish{Chocolate nib buns}
\altdish{Brioche au chocolate et au grués de cacao}
%\serves{}
\makes{12}
\prep{3~hours plus overnight fermentation time}
\source{making-chocolate}

\stage{Dough} %%%%%%%%%%%%%%%%%%%%%%%%%%%%%%%%%%%%%%%%%%%%%%%%%%%

\begin{ingredients}
  \ingr{3}{g}{dry yeast}
  \ingr{42}{g}{sugar}
  \ingrS{110}{g}{warm water}
  \ingr{1}{}{egg}
  \ingrS{110}{g}{cream}
  \ingr{385}{g}{flour}
  \ingr{1.5}{g}{salt}
  \ingrS{}{}{nutmeg, freshly grated}
  \ingr{42}{g}{unsalted butter, melted}
\end{ingredients}

\begin{recipe}
  \begin{enumerate}

  \item Combine yeast, sugar, and warm water.

  \item In a small bowl, whisk egg with cream.

  \item In a separate bowl, whisk together flour, salt, and nutmeg.

  \item In the bowl of a stand mixer, combine yeast mixture with flour
    mixture.  Mix until dough begins to come together, then stream in
    the egg and cream mixture, followed by the melted butter.  Mix
    until the dough is smooth and elastic and pulls away from the
    sides of the bowl, about 6~minutes.

  \item Coat dough with oil.  Bulk ferment, covered and refrigerated,
    overnight (for 8--10~hours).

  \end{enumerate}
\end{recipe}


\stage{Chocolate custard} %%%%%%%%%%%%%%%%%%%%%%%%%%%%%%%%%%%%%%%

\begin{ingredients}
  \ingr{115}{g}{chocolate, 70\%, chopped}
  \ingr{1}{}{egg}
  \ingr{1}{g}{vanilla extract}
  \ingr{150}{g}{whole milk}
  \ingr{}{}{cinnamon, ground (ideally freshly)}
\end{ingredients}

\begin{recipe}
  \begin{enumerate}

  \item Melt the chocolate, set aside.

  \item In another bowl, whisk egg and vanilla just to break up yolk, set aside.

  \item In a sauce pan heat milk and cinnamon just until steaming.
    Make a crème anglaise.  Remove from heat, pour over chocolate,
    whisk thoroughly to combine, and pass through a strainer (to
    remove any coagulated egg) before it cools too much to do so.  (I
    find it more practical to strain the custard onto the chocolate
    rather than to combine and then strain.)  If the custard appears
    chunky, use an immersion blender to emulsify it.

  \item Refrigerate overnight.

  \end{enumerate}
\end{recipe}


\stage{Filling} %%%%%%%%%%%%%%%%%%%%%%%%%%%%%%%%%%%%%%%%%%%%%%%%%

\begin{ingredients}
  \ingr{110}{g}{light brown sugar}
  \ingr{60}{g}{cocoa nibs}
\end{ingredients}

\begin{recipe}
  \begin{enumerate}

  \item Combine and set aside.

  \end{enumerate}
\end{recipe}


\stage{Cinnamon nib sugar} %%%%%%%%%%%%%%%%%%%%%%%%%%%%%%%%%%%%%%%%%%%%%%%%%

\begin{ingredients}
  \ingr{30}{g}{cocal nibs}
  \ingr{200}{g}{sugar}
  \ingr{8}{g}{cinnamon, ground, ideally freshly}
  \ingr{pinch}{}{salt}
\end{ingredients}

\begin{recipe}
  \begin{enumerate}

  \item Place the nibs in a coffee grinder and pulse until fine.  Sift
    through a fine-mesh strainer and combine all ingredients.

  \end{enumerate}
\end{recipe}

Note: This seems like quite a lot of cinnamon.  As the chocolate will
overpower the taste of cinnamon, it is not a typo.


\stage{Assembly} %%%%%%%%%%%%%%%%%%%%%%%%%%%%%%%%%%%%%%%%%%%%%%%%

\begin{ingredients}
  \ingr{}{}{unsalted butter, melted}
\end{ingredients}

\begin{recipe}
  \begin{enumerate}

  \item Remove dough from refrigerator and allow to sit at room
    temperature for at least 20~minutes.

  \item Generously butter  muffin tins.

  \item Turn out dough onto lightly floured surface.  Roll into a
    rectangle, about 30~cm $\times$ 40~cm and 6~mm thick.

  \item Slightly warm the chocolate custard so that it becomes spreadable.
    Spread it on the dough, leaving a 5~mm border at the edges.

  \item Sprinkle with the brown sugar and nib mixture.

  \item Roll from the long edge to form a log.  With a bench knife cut
    into 12~slices.  Place in muffin tins.  Allow to rise for about
    30~minutes.

  \item Preheat oven to 350\degreeC.

  \item Bake buns for 20~minutes or until golden brown.

  \item Allow buns to cool for 10~minutes.  Using a pastry brush, coat
    each roll with melted butter, then roll in the sugar and nib
    mixture, coating the entire surface.  Serve immediately.

  \end{enumerate}
\end{recipe}


%\accord{}
