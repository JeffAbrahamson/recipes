\dish{Tartelettes Croquantes aux Fraises}
\altdish{fraises, tartelettes croquantes}
\altdish{strawberry pie}
\serves{4}
%\makes{}
\prep{}
\source{unknown}


\stage{La p\^ate}

\begin{ingredients}
  \ingr{130}{g}{farine}
  \ingr{1}{pinc\'ee}{levure chimique}
  \ingr{80}{g}{beurre demi-sel, mou}
  \ingr{60}{g}{sucre}
  \ingr{2}{jaunes}{d'oeufs}
\end{ingredients}


\begin{recipe}
  \begin{enumerate}

  \item   Pr\'echauffer le four \`a 180\degreeC\ (th. 6).

  \item M\'elanger la farine, la levure et le sucre.

  \item Ajouter le beurre en pommade puis sabler la p\^ate \`a la
    main.  Fouetter les jaunes d'oeufs et les ajouter.

  \item Bien travailler le tout puis \'etaler la p\^ate au rouleau \`a
    p\^atisserie.  Sur la plaque du four recouverte de papier
    sulfuris\'e, d\'eposer quatre disques de p\^ate r\'ealis\'es \`a
    l?emporte pi\`ece.

  \item Les piquer \`a l'aide d'une fourchette et les cuire au four 10
    minutes.

  \end{enumerate}
\end{recipe}


\stage{La garniture}

\begin{ingredients}
  \ingr{15}{cl}{cr\`eme liquide}
  \ingr{1}{sachet}{sucre vanill\'e}
  \ingr{4}{\cs}{coulis de fraises}
  \ingr{250}{g}{fraises}
  \ingr{}{}{feuilles de menthe}
\end{ingredients}


\begin{recipe}
  \begin{enumerate}

  \item Dans un saladier bien froid, monter la cr\`eme en Chantilly
    avec le sucre vanill\'e.

  \item Ajouter le coulis de fraises.

  \item \'Equeuter les fraises et les couper en lamelles.

  \item Recouvrir de Chantilly les disques de p\^ate et garnir de
    fraises.

  \item La touche du Chef : une feuille de menthe et quelques gouttes
    de coulis pour la d\'ecoration.

  \end{enumerate}
\end{recipe}

\stage{Les astuces}

\begin{enumerate}

\item  On m\'elange dans l'ordre, la farine puis la levure et enfin
  le sucre.
  
\item Sabler la p\^ate = frotter entre les mains le beurre avec la
  farine.
  
\item Fleurer le plan de travail = le saupoudrer de farine pour
  \'eviter que la p\^ate ne colle.
  
\item Pour que le papier sulfuris\'e ne s'envole pas dans le four,
  poser des fourchettes pour faire poids et le maintenir.
  
\end{enumerate}