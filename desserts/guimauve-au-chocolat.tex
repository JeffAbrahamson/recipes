\dish{Guimauve au chocolat}
\altdish{Chocolat, guimauve}
\altdish{Chocolat marshmallow}
%\serves{}
\makes{625 cm$^2$}
\prep{1~hour}
\source{vincent.guerlais}

\begin{ingredients}
  \ingrS{14}{g}{gélatine feuilles}
  \ingrS{90}{g}{blanc d'oeufs}
  \ingr{90}{g}{eau}
  \ingr{40}{g}{miel}
  \ingrS{250}{g}{sucre}
  \ingrS{55}{g}{pâte de cacao}
  \ingr{}{}{poudre de cacao}
\end{ingredients}


\begin{recipe}
  \begin{enumerate}

  \item Préparer une plaque bien saupoudré de poudre de cacao. Mettre
    une cadre dessus, environ 25~cm$\,\times\,$25~cm.

  \item Laisser ramollir la gélatine dans de l'eau froide.

  \item Commencer à monter les blanc d'oeufs.

  \item Faire cuire le miel, le sucre, et l'eau à 130\C.

  \item Ajouter la gélatine au sirop, puis verser dans les blancs
    d'oeufs semi-montés, puis laisser tiédir au batteur jusqu'à 45\C\,
    environ. (Essentiellement, faire une meringue italienne.)

  \item Ajouter le cacao fondu et cadrer. Saupoudrer de poudre de
    cacao. Laisser geler quelques heures au moins à température
    ambiante.

  \end{enumerate}
\end{recipe}

La recette originelle ne propose que 35~g de pâte de cacao.

%\accord{}
