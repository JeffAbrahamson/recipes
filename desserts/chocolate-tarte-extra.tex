\dish{Tarte Extraordinairement Chocolat}
\altdish{Chocolate Tarte, extraordinaire}
%\serves{12}
\makes{12}
\prep{repos de la p\^ate + 2 heures + refroidissement}
\source{valrhona}

\textbf{P\^ate sabl\'ee amandes}

\begin{ingredients}
  \ingr{120}{g}{beurre}
  \ingr{2}{g}{sel fin}
  \ingr{80}{g}{sucre glace}
  \ingr{100}{g}{poudre d'amande}
  \ingr{1}{}{oeuf}
  \ingr{60}{g}{farine}
  \ingr{180}{g}{farine}
\end{ingredients}


\begin{recipe}
  \begin{enumerate}

  \item Proc\'eder \`a un premier m\'elange avec tous les
    ingr\'edients sauf les 180~g de farine.

  \item D\`es que le m\'elange est homog\`ene, ajouter les 180~g de
    farine restant, ceci de fa\c{c}on tr\`es br\`eve.

  \item Faire douze petits ronds applatis, emballer en film
    \'etirable, et laisser reposer au refrigerateur.

  \item Etaler chaque disque entre deux feuilles de papier cuisson et
    mettre dans des moules \`a tarte individuelles.

  \item Faire cuire 10~minutes \`a 155--160\degreeC.

  \end{enumerate}
\end{recipe}

\ \vbox{
\textbf{Biscuit chocolat}

\begin{ingredients}
  \ingr{60}{g}{chocolat couverture cara\"ibe 66\%}
  \ingr{20}{g}{beurre}
  \ingr{2}{}{blancs d'oeuf}
  \ingr{20}{g}{sucre semoule}
  \ingr{2}{}{jaunes d'oeuf}
\end{ingredients}
}

\begin{recipe}
  \begin{enumerate}

  \item Faire fondre le chocolat et le beurre \`a 40--45\degreeC\ environ.

  \item Monter le blanc avec le sucre.  M\'elanger au fouet le jaune.
    Puis, \`a la spatule, incorporer le chocolat et le beurre fondus.

  \item Verser sur les fonds de tartes cuites et faire cuire 10--15
    minutes \`a 180\degreeC.

  \end{enumerate}
\end{recipe}


\textbf{Ganache}

\begin{ingredients}
  \ingr{400}{g}{cr\`eme}
  \ingr{360}{g}{chocolat couverture cara\"ibe 66\%}
  \ingr{60}{g}{beurre}
\end{ingredients}


\begin{recipe}
  \begin{enumerate}

  \item Faire bouillir la cr\`eme.  Verser un tier sur le chocolat.
    Remuer avec une spatule pour que le chocolat fonde (plus ou moins).

  \item Proc\'eder de la m\^eme mani\`ere que lorsqu'on fait une
    mayonnaise.  C'est \`a dire, m\'elanger \'energiquement \`a l'aide
    d'une spatule de fa\c{c}on \`a cr\'eer un noyau \'elastique et
    brillant, puis rajouter petit \`a petit la cr\`eme comme on ajoute
    de l'huile \`a une mayonnaise.  La texture devra \^etre
    conserv\'ee jusqu'en fin de m\'elange.

  \item Ajouter enfin le beurre en morceau.

  \item Verser sur les fonds de tartes refroidis.  Laisser refroidir.

  \end{enumerate}
\end{recipe}
