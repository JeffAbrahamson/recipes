\dish{あんパン}
\altdish{餡パン}
\alsodish{anpan}
%\serves{}
\makes{8 breads}
\prep{30 minutes + 2 hours}
\source{japanese-cooking-101}

\begin{ingredients}
  \ingr{200}{ml}{whole milk}
  \ingr{5}{g}{dry yeast}
  \ingrS{30}{g}{sugar}
  \ingr{240}{g}{bread flour}
  \ingr{60}{g}{cake flour}
  \ingr{5}{g}{salt}
  \ingrS{10}{g}{dry milk powder}
  \ingrS{30}{g}{butter, room temperature}
  \ingr{280}{g}{\hyperref[あんこ]{あんこ}}
\end{ingredients}


\begin{recipe}
  \begin{enumerate}

  \item Mix milk, yeast, and sugar in stand mixer bowl.  Let sit for 5~minutes.
    
  \item Combine remaining ingredients except butter.  Mix until dough becomes a ball.

  \item Add butter and continue mixing.
    
  \item Cover with plastic and let rise about one hour.
    
  \item Make balls of anko, 35~g each (about 4~cm diameter)
    
  \item Degas the dough and cut into 8~equal weight pieces (about 70~g
    each).  Form balls and allow to rest 15~minutes.
    
  \item Flatten each ball of dough with a rolling pin to form a 10~cm
    disk.  Put a ball on anko in the center of the round, wrap with
    dough, and pinch the ends of the dough so that the anko is sealed
    inside.
    
  \item Place on a baking sheet lined with parchment paper, leaving
    5~cm between the balls.  Let rise until doubled, about one hour.
    
  \item Brush with egg wash (egg with a pinch of salt), sprinkle with
    black sesame seeds.  Bake in preheated oven at 205\degreeC for
    about 10~minutes until golden brown.


  \end{enumerate}
\end{recipe}

%\accord{}
