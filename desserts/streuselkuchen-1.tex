\dish{Streuselkuchen}
%\altdish{}
\serves{15--25}
%\makes{}
\prep{1h sur 2h}
\source{internet}

\stage{Teig}

\begin{ingredients}
  \ingr{200}{ml}{lait}
  \ingr{10}{g}{levure de boulanger}
  \ingrS{10}{g}{sucre vanillé}
  \ingr{375}{g}{farine}
  \ingr{60}{g}{beurre}
  \ingr{10}{g}{sel}
  \ingr{40}{g}{sucre}
  \ingr{1}{}{oeuf}
\end{ingredients}



\begin{recipe}
  \begin{enumerate}

  \item Mélanger lait, levure, et sucre vanillé plus 100~g de la
    farine.  Laisser fermenter 15--20 minutes.
    
  \item Ajouter les ingrédients restants.  Pétrir.
    
  \item Laisser fermenter 1~heure.

  \end{enumerate}
\end{recipe}

\stage{Streusel}

\begin{ingredients}
  \ingr{300}{g}{farine}
  \ingr{175}{g}{beurre}
  \ingrS{150}{g}{sucre}
  \ingr{25}{g}{beurre fondu (pour la pâte)}
\end{ingredients}


\begin{recipe}
  \begin{enumerate}

  \item Pendant que la pâte fermente, mélanger farine, beurre, et sucre.
    
  \item Beurrer un moule à charnière de 25~cm de diamètre.
    
  \item Dégazer la pâte sur une surface légèrement fariner.
    Transférer au moule et étaler contre les parois.
    
  \item Avec un pinceau, peindre la pâte du beurre fondu.  Puis verser le streusel.
    
  \item Laisser apprêter 30 minutes.  Préchauffer le four à 200\degreeC.
    
  \item Enfourner, laisser cuire 25 minutes.
    
  \item Laisser refroidir 30 minutes sur une volette avant de démouler.

  \end{enumerate}
\end{recipe}


%\accord{}
