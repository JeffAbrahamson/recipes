\dish{Galette des rois au chocolat}
%\altdish{}
\serves{8}
%\makes{}
\prep{3~heures sur 6--8 heures}
\sourceD{unknown}{dit de Pierre Hermé}


\stage{Beurre manié}

\begin{ingredients}
  \ingr{250}{g}{beurre}
  \ingr{100}{g}{farine}
\end{ingredients}


\begin{recipe}
  \begin{enumerate}

  \item Dans le bol d’un robot muni de la feuille, ramollir le
    beurre.  Ajouter la farine et mélanger le moins possible jusqu’à ce
    que le mélange soit homogène. Etaler en rectangle sur une feuille
    de papier siliconé, couvrir d’une deuxième feuille et mettre au
    réfrigérateur pendant 1~heure.

  \end{enumerate}
\end{recipe}


\stage{Détrempe}

\begin{ingredients}
  \ingr{100}{g}{d'eau}
  \ingr{1.5}{g}{vinaigre blanc}
  \ingr{10}{g}{ fleur de sel}
  \ingr{235}{g}{farine}
  \ingr{75}{g}{beurre}
\end{ingredients}


\begin{recipe}
  \begin{enumerate}

  \item Faire ramollir le beurre au four micro-ondes afin de le rendre
    pommade.

  \item Dans le bol d’un robot muni du crochet, faire la détrempe en
    mélangeant les ingrédients.

  \item Mettre en carré sur une plaque recouverte d’une feuille de
    papier siliconé, filmer et laisser reposer au réfrigérateur
    pendant 1~heure.

  \end{enumerate}
\end{recipe}


\stage{Pâte feuilletée inversée}

\begin{recipe}
  \begin{enumerate}

  \item Enchâsser la détrempe (2) dans le Beurre manié (1), les deux
    préparations doivent avoir une texture identique. Étaler la pâte
    en long et donnez 2~tours doubles à 2~heures d'intervalle en
    mettant la pâte au réfrigérateur entre chaque tour. Puis 1~tour
    simple avant de détailler. On peut stocker la pâte feuilletée
    plusieurs jours au réfrigérateur à deux tours doubles.

  \end{enumerate}
\end{recipe}


\stage{Crème pâtissière}

\begin{ingredients}
  \ingr{100}{g}{lait frais entier}
  \ingr{1}{}{jaune d'oeuf}
  \ingr{15}{g}{sucre semoule}
  \ingr{5}{g}{poudre à crème \textit{(seulement l'originelle)}}
  \ingr{5}{g}{farine}
  \ingr{10}{g}{beurre}
  \ingr{20}{g}{chocolat (coeur de guanaja) à 30\C{} \textit{(pas dans l'originelle)}}
  \ingr{20}{g}{poudre de cacao \textit{(pas dans l'originelle)}}
\end{ingredients}


\begin{recipe}
  \begin{enumerate}

  \item Tamiser ensemble la farine \textit{et le poudre à crème} et mélanger le
    jaune d’oeuf.

  \item Faire chauffer le lait et le sucre semoule et verser sur le
    premier mélange.

  \item Remettre l’ensemble dans la casserole.  Porter la crème
    pâtissière à 82\C{} \textit{(originelle : à ébullition et laisser cuire pendant 5 minutes en
      mélangeant vivement à l’aide d’un fouet).}

  \item Faire refroidir, à environ 50\C. Ajouter le beurre, chocolat,
    et poudre de cacao; mélanger et faire refroidir complètement au
    réfrigérateur avant utilisation.

  \end{enumerate}
\end{recipe}

\stage{Crème d’amande}

\begin{ingredients}
  \ingr{100}{g}{beurre}
  \ingr{100}{g}{sucre glace}
  \ingr{100}{g}{poudre d'amande blanche}
  \ingr{1}{}{oeuf}
  \ingr{10}{g}{crème \textit{(originelle : poudre à crème)}}
  \ingr{120}{g}{crème patissière}
  \ingr{10}{g}{grand marnier \textit{(originelle : rhum brun agricole (Clément))}}
  \ingr{10}{g}{poudre de cacao \textit{(pas dans l'originelle)}}
\end{ingredients}


\begin{recipe}
  \begin{enumerate}

  \item Dans le bol d’un robot muni de la feuille ou dans une terrine,
    malaxer le beurre sans le faire foisonner, puis ajouter tous les
    ingrédients un à un en continuant de mélanger à petite vitesse.
    Utiliser aussitôt ou réserver au réfrigérateur.

  \end{enumerate}
\end{recipe}

N.B. : le beurre ainsi que l'ensemble de la préparation ne doivent pas
être foisonnée, sinon, lors de la cuisson, la crème d'amandes gonfle
et retombe aussitôt de façon irrégulière.


\stage{Dorure}

\begin{ingredients}
  \ingr{1}{}{oeuf}
  \ingr{\fracH}{}{jaune d'oeuf}
  \ingr{1}{}{pincée sel}
\end{ingredients}


\begin{recipe}
  \begin{enumerate}

  \item Dans un bol fouetter avec le pinceau à pâtisserie l’oeuf
    entier avec le demi-jaune et le sel.

  \end{enumerate}
\end{recipe}


\stage{Montage}

\begin{recipe}
  \begin{enumerate}

  \item Couper la pâte en deux. Etaler chaque pâton sur le plan de
    travail fariné en forme de carré de 2~mm d’épaisseur.  À l’aide
    d’un petit couteau planté droit dans la pâte, découper dans chacun
    un disque de 28~cm de diamètre.  Balayer la pâte pour retirer
    l’excédent de farine.

  \item Retourner sur l’autre face (le dessous d’une abaisse de pâte
    est toujours plus lisse est parfait que le dessus) un premier
    disque de pâte sur la plaque à pâtisserie recouverte de papier
    siliconé.  Dessiner légèrement sur la pâte un cercle à 3 cm des
    bords du disque afin de délimiter la surface sur laquelle la crème
    d’amande va s'étaler.  À l’aide d’un pinceau trempé dans de l’eau
    froide badigeonner la pâte, à 1~cm du bord du disque afin d’éviter
    que l’eau coule sur les côtés.  Etaler la crème d’amande et
    égaliser le dessus avec le dos d’une cuillère.  Enfouisser la fève
    à 1~cm des bords de la crème d’amande.

  \item Retourner sur l’autre face le second disque de pâte.  Le poser
    exactement sur le premier où la crème d’amande est étalée.  Du
    bout des doigts appuyer sur les bords des deux disques de pâte
    afin qu’ils soient parfaitement scellés. Mettre la plaque
    30~minutes au réfrigérateur.

  \item A l’aide de la pointe d’un petit couteau placé à l’envers
    (lame vers vous) et tenu dans le sens du biais, soulever les deux
    bords joints de la galette et les festonner en enfonçant
    légèrement le couteau tous les centimètres tout en appuyant sur la
    pâte avec l’index entre chaque entaille.

  \item Avec la dorure et un pinceau, badigeonner en toute la surface
    de la galette en prenant soin que la dorure ne coule pas sur les
    bords.  Décorer le dessus de la galette en rayant uniformément la
    pâte avec la pointe du couteau tenu à l’envers, la strier en
    partant du centre en dessinant des arcs de cercle espacés tous les
    2~cm.

  \item Glisser la galette dans le four préchauffé à 230\C, baissé
    aussitôt à 190\C. Laisser cuire 45~minutes, pas moins.

  \item La Servir de préférence tiède.

  \end{enumerate}
\end{recipe}

%\accord{}
