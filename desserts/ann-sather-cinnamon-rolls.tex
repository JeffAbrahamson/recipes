\dish{Ann Sather's Cinnamon Rolls}
\altdish{Cinnamon Rolls, Ann Sather's}
%\serves{}
\makes{18}
\prep{1 hour}
\source{unknown}

\textbf{Dough:}

\begin{ingredients}
  \ingr{1}{c}{milk}
  \ingr{1}{Tbsp}{dried yeast}
  \ingr{\fracQ}{c}{warm water}
  \ingr{\fracT}{c}{sugar plus 1 tsp}
  \ingr{4}{Tbsp}{butter, melted}
  \ingr{1\fracH}{tsp}{salt}
  \ingr{2\fracH--3}{c}{flour}
\end{ingredients}


\begin{recipe}
  \begin{enumerate}

  \item Scald milk and let cool.

  \item In a large bowl, stir the yeast and 1 tsp sugar into the warm
    water and let stand for 5 minutes.

  \item Stir in the cooled milk, melted butter, salt, and 1 cup of
    flour.  Beat until smooth.

  \item Gradually stir in remaining flour, keeping the dough smooth.
    If the dough is still moist, stir in more flour, 1 Tbsp at a time,
    to make a soft dough.

  \item Cover with a dry cloth and let rise in a warm place until
    dough doubles in bulk, about 1 hour.

  \end{enumerate}
\end{recipe}


\textbf{Filling:}

\begin{ingredients}
  \ingr{4}{Tbsp}{butter, softened}
  \ingr{\fracH}{c}{brown sugar}
  \ingr{1}{Tbsp}{ground cinnamon}
\end{ingredients}


\begin{recipe}
  \begin{enumerate}

  \item Combine cinnamon and sugar.

  \item Butter and flour a baking pan (or two?).

  \item Punch down dough and divide in half.  On a lightly oiled
    board, roll out each piece to a $12\times 18$ inch rectangle.
    Spread butter, then cinnamon and sugar on top.  Beginning with the
    long side, roll up tightly, jelly-roll fashion.  Cut each roll in
    nine, placing on the prepared pan.

  \item Cover and let rise until doubled, about 45 minutes.

  \item Preheat oven to 350\F.

  \item Bake 12--15 minutes or until golden brown.

  \end{enumerate}
\end{recipe}


\textbf{Glaze:}

\begin{ingredients}
  \ingr{\fracH}{c}{powdered sugar}
  \ingr{4}{Tbsp}{butter}
  \ingr{1}{tsp}{vanilla extract}
\end{ingredients}


\begin{recipe}
  \begin{enumerate}

  \item In a small bowl beat the glaze ingredients until creamy and
    smooth.

  \item On removing rolls from oven, place on wire racks to cool.
    While still hot, coat with glaze.  Allow to cool.

  \end{enumerate}
\end{recipe}

The page from which this recipe was taken indicated that the recipe
came from \textit{Ann Sather: A Chicago Tradition} by Ann Sather.
