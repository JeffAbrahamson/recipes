\dish{Kouign Amann}
%\altdish{}
\serves{20}
%\makes{}
\prep{1 heure (?)}
\source{maison-kouign-amann}

C'est une sp\'ecialit\'e de Douarnenez, quoique certaines de sources
parlent de Scaer, et un excellent gateau quand il est r\'eussi.

\begin{ingredients}
  \ingr{600}{g}{farine}
  \ingr{500}{g}{beurre}
  \ingr{500}{g}{sucre}
  \ingr{2--3}{dl}{d'eau}
  \ingr{1}{pinc\'ee}{levure de bi\`ere}
  \ingr{1}{pinc\'ee}{sel}
\end{ingredients}


\begin{recipe}
  \begin{enumerate}

  \item Tamiser la farine, la mettre sur une planche, faire une
    fontaine, y mettre le sel, la levure, l'eau tr\`es froide.

  \item Avec le bout des doigts, m\'elanger le tout
    rapidement. Envelopper le p\^aton d'un linge, laisser reposer cinq
    minutes.

  \item Pendant ce temps, mettre le beurre dur dans le coin d'un
    torchon mouill\'e, replier le torchon et triturer le beurre pour lui
    donner une certaine souplesse.

  \item Etaler la p\^ate, mettre le beurre, et le sucre au milieu,
    replier les bords comme une p\^ate feuillet\'ee et plier deux fois en
    quatre.

  \item Mettre aussit\^ot \`a four chaud (200\C) une demi-heure.

  \end{enumerate}

  \textbf{Conseil}

  La temp\'erature du beurre et de la pi\`ece o\`u se pr\'epare le gateau est
  importante. Certains p\^atissiers laissent la p\^ate reposer dans un
  torchon, avant de faire les tours. D'autres utilisent de l'eau tr\`es
  froide pour faire la p\^aton, d'autres encore de la levure dilu\'ee \`a
  l'eau ti\`ede ou lavent le g\^ateau au lait avant la cuisson.

  Ne l\'esinez pas sur le beurre, et une fois le g\^ateau cuit, arrosez le
  du beurre fondu qui n'a pas p\'en\'etr\'e dans la p\^ate. Laisser refroidir
  dans le moule (en fer) de cuisson pour ne pas vous br\^uler avec le
  caramel.

  Pour r\'echauffer le g\^ateau, faites chauffer une po\`ele, la retirer
  quand elle est chaude. D\'emouler le g\^ateau, le glisser dans la
  po\`ele. Couvrez, patientez 5 \`a 8 mn hors du feu. R\'egalez vous !

\end{recipe}
