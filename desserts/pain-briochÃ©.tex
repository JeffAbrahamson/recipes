\dish{Pain brioché}
\altdish{brioché, pain}
%\serves{}
%\makes{}
\prep{4h40}
\source{paul.magnette}

\begin{ingredients}
  \ingr{60}{ml}{lait entier}
  \ingr{155}{ml}{eau tiède}
  \ingr{100}{g}{levain liquide (froment, T65, 100 \% hydration)}
  \ingr{3}{g}{levure sèche (ou 6 g fraîche)}
  \ingr{2}{}{oeufs}
  \ingr{25}{g}{sucre}
  \ingrS{500}{g}{farine T65}
  \ingrS{10}{g}{sel}
  \ingr{125}{g}{beurre doux}
\end{ingredients}


\begin{recipe}
  \begin{enumerate}

  \item Dans la cuve du robot verser lait, eau, levain liquide,
    levure, oeufs battus, sucre, et farine.  Pétrir en première
    vitesse pendant 2--3~minutes.
    
  \item Intégrer le sel, puis le beurre bien froid en petits morceaux
    ajoutés progressivement.  Pétrir 3--4~minutes en deuxième vitesse
    jusqu'à ce que la pâte soit lisse.
    
  \item Couvrir d'un linge humide et laisser reposer 1h30 à
    température ambiante.
    
  \item Sortir la pâte de la cuve à l'aide d'une corne, la déposer sur
    un plan de travail bien fariné et façonner délicatement le pâton
    du bout des doigts pour lui donner une forme allongée de la taille
    du moule.  Le déposer dans le moule, couvrir d'un linge humide et
    laisser reposer 1~h à température ambiante.
    
  \item Enfourner à 240\degreeC{} (éventuellement avec de la buée en
    début de cuisson---un verre d'eau jeté sur la lèchefrite placée au
    fond du four).  Après 20~minutes ouvrir la porte pour laisser
    évacuer l'humidité et cuire encore 20~minutes à 210\degreeC.
    
  \item Sortir le pain du moule et le laisser ressuer sur une volette
    1~h avant de le manger.
    
  \item Le pain se mange le jour même, sinon trancher et congeler ou
    bien prévoir toasté.

  \end{enumerate}
\end{recipe}

%\accord{}
