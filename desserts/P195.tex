\dish{P195}
\altdish{cookies au chocolat Coeur de Guanaja}
%\serves{}
\makes{16}
\prep{30 minutes + 2 heures de repos + 10 minutes de cuisson}
\source{cookie-de-nos-reves}

\begin{displaymath}
  195 = 125 + 70
\end{displaymath}

\begin{ingredients}
  \ingr{250}{g}{beurre salé, en pommade}
  \ingr{100}{g}{chocolat Guanaja (70 \%), à 30 \degreeC}
  \ingrS{110}{g}{cassonade}
  \ingr{60}{g}{oeuf}
  \ingr{10}{g}{levure chimique}
  \ingr{100}{g}{farine, T45}
  \ingr{100}{g}{farine, T55}
  \ingr{120}{g}{chocolat Coeur de Guanaja P125 (80 \%)}
\end{ingredients}


\begin{recipe}
  \begin{enumerate}

  \item Faire fondre le chocolat Guanaja au  bain-marie, apporter à 30\degreeC{}.

  \item Mélanger avec le beurre pommade pour obtenir un beurre pommade
    au chocolat.  Éviter surtout que le beurre ne fonde pas.

  \item Fouetter avec le cassonade jusqu'à blanchi.

  \item Ajouter l'oeuf, puis farine et levure.

  \item Ajouter les fèves de P125 non concassées et mélanger jusqu'à
    l'obtention d'une pâte homogène.

  \item Avec un cadre, former la pâte en carré à un épaisseur de
    4~cm.  Laisser reposer au moins 2~heures au réfrigérateur.

  \item Préchauffer le four à 180--190\degreeC{} en mode chaleur
    tournante.  Détailler la pâte en 16 cubes de 50~g (approx).

  \item Cuire 10~minutes.  Bien laisser refroidir avant de les
    manipuler car le coeur est très fondant.

  \end{enumerate}
\end{recipe}

%\accord{}
