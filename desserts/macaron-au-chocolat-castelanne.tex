\dish{Macaron au Chocolat de M. Castelanne}
\altdish{Chocolat, Macaron, Castelanne}
%\serves{}
\fait{environ 70 coques}
\prep{2 heures}
\source{philippe.castelanne}

\stage{Coques}

\begin{ingredients}
  \ingr{150}{g}{sucre (pour la meringue)}
  \ingr{60}{ml}{d'eau}
  \ingr{60}{g}{blanc d'oeufs (environ 2 oeufs, pour la meringue)}
  \ingr{150}{g}{sucre (pour le poudre d'amande)}
  \ingr{110}{g}{poudre d'amande}
  \ingr{30}{g}{poudre de cacao}
  \ingr{60}{g}{blanc d'oeufs (environ 2 oeufs)}
\end{ingredients}


\begin{recipe}
  \begin{enumerate}

  \item Meringue italienne:  Mettre le sucre et l'eau dans
    une casserole et faire cuire jusqu'\`a 118\degreeC.  Battre les blancs d'oeufs
    en neige, puis \`a vitesse moyenne verser le sucre cuit en mince filet
    sur les blancs d\'ej\`a mousseux.  Laisser refroidir toujours en
    battant.  Quand la mousse prend corps, augmenter la vitesse au
    maximum pour serrer les blancs.

  \item Pendant que le sucre cuit et puis que la meringue refroidisse,
    passer le sucre pour le poudre d'amende au blender afin de cr\'eer
    du sucre glace (donc sans amidon comme dans le sucre glace
    commercial).  Ajouter le poudre d'amende pour avoir un poudre plus
    fin, mais attention \`a ne pas faire exprimer l'huile.  Ajouter le cacao.

  \item Ajouter le blanc d'oeufs non-battu au poudre.  Puis peu \`a
    peu ajouter la meringue au poudre en macaronnant \`a la maryse.
    Le m\'elange doit briller et faire ruban.  Une trace de doigt d'un
    centimetre doit refermer lentement.

  \item Remplir une poche \`a douille (douille de 10 mm).  Dresser sur
    une feuille de papier cuisson sur une plaque.  Les claquer, puis
    les laisser cro\^uter 30~minutes.

  \item Pr\'echauffer le four \'a 160\degreeC\ (ventil\'e) ou 180\degreeC\
    (statique).  Faire cuire les coques de 10--11~minutes.

  \item A la sortie du four, renverser les coques sur des volettes.

  \end{enumerate}
\end{recipe}


\stage{Ganache}

\begin{ingredients}
  \ingr{100}{g}{chocolat}
  \ingr{90}{g}{cr\`eme fleurette}
  \ingr{25}{g}{beurre}
\end{ingredients}


\begin{recipe}
  \begin{enumerate}

  \item Hacher le chocolat.  Chauffer la cr\`eme \`a 40\degreeC\ et verser
    sur le chocolat.

  \item A 30\degreeC\ mettre le beurre pommade et l'incorporer.

  \item Laisser refroidir.  Remplir une poche \`a douille et dresser
    et marier les coques.

  \item Laisser les macarons, maintenant complets, reposer un jour pour
    qu'ils prennent l'humidit\'e.

  \end{enumerate}
\end{recipe}

\textbf{Notes :}
\begin{itemize}
\item Pour des macarons non-chocolat, c'est 150~g de poudre d'amande
  au lieu de 110~g poudre d'amande plus 30~g de poudre de cacao.
\item Une ganache au fruit peut se faire en m\'elangeant 100~g de
  confiserie blanche, 10~g de beurre de cacao, 60~g de pulpe de fruit,
  et 30~g de beurre.
\end{itemize}
