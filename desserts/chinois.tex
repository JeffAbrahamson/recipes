\dish{Chinois}
\altdish{rosekueche}
\altdish{schneckekueche}
%\serves{}
%\makes{}
\prep{}
\sourceD{doerflinger}{avec modifications de Théo Birklé}

\textit{(Aussi schneckekueche ou rosekueche)}

\begin{ingredients}
  \ingr{20}{g}{levure (en pâte)}
  \ingr{250}{ml}{lait}
  \ingr{500}{g}{farine}
  \ingr{125}{g}{beurre}
  \ingr{5}{g}{sel}
  \ingrS{80}{g}{sucre}

  \ingr{}{}{cannelle en poudre}
  \ingr{200}{g}{amandes en poudre}
  \ingr{200}{g}{raisins secs}
  \ingrS{75}{g}{sucre en poudre}

  \ingrS{1}{}{blanc d'oeuf}

  \ingr{}{}{sucre glace}
  \ingr{}{}{petit verre de kirsch}
\end{ingredients}


\begin{recipe}
  \begin{enumerate}

  \item Préparer la pâte comme une pâte levée.

  \item Faire une abaisse de 4~cm d'épaisseur et la découper en
    6~bandes (selon la taille de la tourtière).  Les badigeonner d'un
    peu de beurre fondu, les saupoudrer de sucre, de poudre d'amandes,
    de cannelle, et de raisins secs.

  \item Rouler les bandes de pâtes et les dresser sur une tourtière,
    en mettant un ``escargot'' au centre.  Laisser lever 30~minutes.
    Dorer à l'oeuf et faire cuire à four moyen environ 45~minutes.
    Après cuisson recouvrir avec le sucre glace imprégné du kirsch.

  \end{enumerate}
\end{recipe}

%\accord{}

{\it Recette originelle : Doerflinger propose 500~g de farine, 100~g
  d'amandes, et 100~g de raisins secs.  Elle ne met pas sucre en
  poudre avec la cannelle.  }
