\dish{Tartare de tomates vertes, soupe de tomates jaunes, et glaçon de tomates rouges}
\altdish{Tomates : tartare, soupe, et glaçon}
\altdish{Yellow tomato soup}
\altdish{Green tomato tartare and its soup}
\altdish{Glaçon de tomates rouges : tartare et soup}
\serves{4}
%\makes{}
\prep{30 minutes (plus 2h réfrigération plus 5m cuisson)}
\source{cvf-134}

\begin{ingredients}
  \ingr{150}{g}{tomates vertes (variété ``green zebra'' si possible)}
  \ingr{400}{g}{tomates jaunes (variété ``ananas'' si possible)}
  \ingr{150}{g}{tomates rouges, pelées}
  \ingr{50}{g}{tomates confites}
  \ingr{jus}{d'un}{citron}
  \ingr{2}{}{brins de thym}
  \ingr{1}{}{gousse d'ail}
  \ingr{1}{\cs}{huile d'olive}
  \ingr{}{}{Tabasco (quelques gouttes)}
  \ingr{}{}{sel}
  \ingr{}{}{poivre}
\end{ingredients}


\begin{recipe}
  \begin{enumerate}

  \item Laver et essuyer les tomates.  Après avoir ôté les pédoncules,
    les couper séparément : les vertes en dés, les rouges en morceaux.
    Eplucher l'ail.  Laisser les jaunes entières.

  \item \textbf{(2h en avance)} Mettre les morceau de tomates rouges,
    les tomates confites, et quelques gouttes de Tabasco dans un
    blender.  Saler, poivrer, et faire fonctionner l'appareil pour
    obtenir un coulis.  Verser le coulis dans un bac à glaçons et
    réserver deux heures au congélateur.

  \item \textbf{(Préparer la soupe)} Mettre les tomates jaunes dans
    une casserole avec 15~cl d'eau, l'ail, et 1~brin de thym.  Saler,
    porter à ébulition pendant 20~minutes.  Retirer l'ail et le thym,
    peler les tomates, et verser dans le blender.  Ajouter la moitié
    du jus de citron et faire fonctionner l'appareil pour obtenir une
    soupe lisse.  Pouvrer et réserver dans la casserole sur feu doux.

  \item \textbf{(Préparer le tartare)} Effeuiller et hacher l'autre
    brin de thym.  Assaisonner les tomates vertes de sel, poivre,
    huile d'olive, thym, et du reste du jus de citron.  Mélanger et
    réserver à température ambiante.

  \item \textbf{(Au moment de servir)} Répartir le tartare en dôme au
    centre de quatre assiettes creuses.  (Utiliser des cercles si
    possible.)  Poser un glaçon de tomate rouge sur chaque tartare,
    versez la soupe jaune autour.  Accompagner éventuellement chaque
    tartare d'une tartine de pain de campagne toastée, frottée d'ail,
    badigeonnée d'huile d'olive, et saupoudrée de parmesan
    fraîchement râpé.

  \end{enumerate}
\end{recipe}

\accord{un viognier de l'Ardèche (blanc)}
