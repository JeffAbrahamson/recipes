\dish{Winter Squash Soup with Fried Sage Leaves}
\altdish{Squash (winter) soup with fried sage leaves}
\altdish{Butternut (winter) squash soup with fried sage leaves}
\altdish{Roasted Butternut (winter) squash soup with fried sage leaves}
\altdish{Winter squash soup with fried sage leaves}
\altdish{Fried sage leaves, in winter squash soup}
\altdish{Sage leaves, fried, in winter squash soup}
\serves{4--6}
\prep{1 hr(?)}
\source{madison.VCfE}

\begin{ingredients}
  \ingr{2\fracH--3}{lbs}{winter squash}
  \ingr{\fracQ}{c}{olive oil, plus extra for the squash}
  \ingr{6}{}{garlic cloves, peeled}
  \ingr{12}{}{sage leaves, whole, plus 2 Tbsp chopped}
  \ingr{2}{}{onions, finely chopped}
  \ingr{}{}{leaves from 4 thyme sprigs or \fracQ tsp dried}
  \ingr{\fracQ}{c}{chopped parsley}
  \ingr{}{}{salt and pepper}
  \ingr{2}{qts}{water or stock}
  \ingr{\fracH}{c}{Fontina, pecorino, or ricotta salata, diced into small cubes}
\end{ingredients}


\begin{recipe}
  \begin{enumerate}

  \item Preheat the oven to 375\F.  Halve the squash and scoop out the
    seeds.  Brush the surfaces with oil, stuff the cavities with the
    garlic, and place them cut side down on a baking sheet.  Bake
    until tender when pressed with a finger, about 30 minutes.

  \item Meanwhile, in a small skillet, heat the \fracQ cup oil until
    nearly smoking, then drop in the whole sage leaves and fry until
    speckled and dark, about 1 minute.  (They will burn 15 seconds or
    so later if not removed promptly!)  Set the leaves aside on a
    paper towel and transfer the oil to a wide soup pot.

  \item Add the onions, chopped sage, thyme, and parsley and cook over
    medium heat until the onions have begun to brown around the edges,
    12 to 15 minutes.  Scoop the squash flesh into the pot along with
    any juices that have accumulated in the pan.  Peel the garlic and
    add it to the pot along with 1\fracH tsp salt and the water and
    bring to a boil.  Lower the heat and simmer, partially covered,
    for 25 minutes.  If the soup becomes too thick, simply add more
    water to thin it out.  Taste for salt.

  \item Depending on the type of squash you've used, the soup will be
    smooth or rough.  Puree or pass it through a food mill if you want
    a more refined soup.  Ladle it into bowls and distribute the
    cheese over the top.  Garnish each bowl with the fried sage
    leaves, add pepper, and serve.

  \end{enumerate}
\end{recipe}
