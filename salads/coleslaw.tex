\dish{Coleslaw}
\altdish{Patrick Roger's coleslaw}
%\pour{}
%\fait{}
\prep{30 minutes}
\source{patrick:rogers}

\begin{ingredients}
  \ingr{\fracH}{}{cabbage}
  \ingr{1}{}{apple}
  \ingr{1}{}{lemon (juice of)}
  \ingr{1}{clove}{garlic}
  \ingr{1}{large}{shallot}
  \ingr{}{fat}{carrot}
  \ingr{\fracH}{\tbsp}{Dijon mustard}
  \ingr{\fracQ}{c}{olive oil}
  \ingr{}{}{Tabasco to taste}
  \ingr{1}{\tsp}{cumin}
  \ingr{4--5}{\tsp}{sugar}
  \ingr{}{}{salt, to taste}
  \ingr{}{}{black pepper, to taste}
  \ingr{}{}{herbes de provence, to taste}
\end{ingredients}


\begin{recipe}
  \begin{enumerate}

  \item Raper le choux, mélanger tout, et laisser reposer au réfrigérateur au moins un jour.

  \end{enumerate}
\end{recipe}

\textbf{Variations:}

\begin{itemize}
\item Add fine slices of red bell pepper.
\item Replace olive oil with fat free sour cream or fromage blanc.
\item If your cabbage is particularly dry or tough, shred it and soak
  it in apple cider vinegar and some sugar overnight. Drain it before
  adding it to the rest of the ingredients.
\end{itemize}


%\accord{}
