\dish{Tartes Fine aux Tomates}
\altdish{Tomates, tartes fines}
\serves{4}
%\makes{}
\prep{10 minutes + 25 minutes de cuisson}
\source{unknown}

\begin{ingredients}
  \ingr{4}{}{tomates}
  \ingr{2}{}{rouleaux de p\^ate sabl\'ee}
  \ingr{2}{\cs}{moutarde}
  \ingr{}{}{basilic}
  \ingr{}{}{huile d'olive}
  \ingr{}{}{sel}
  \ingr{}{}{poivre}
  \ingr{}{}{parmesan}
  \ingr{}{}{vinaigre balsamique}
\end{ingredients}


\begin{recipe}
  \begin{enumerate}

  \item Pr\'echauffer le four \`a 210\degreeC\ (th. 7).

  \item D\'ecouper 4 disques de p\^ate, les piquer et les cuire au
    four pendant 10 mn environ.

  \item Badigeonner de moutarde les p\^ates pr\'ecuites puis les
    recouvrir de rondelles de tomates. Ajouter le basilic cisel\'e.


  \item Arroser d'un filet d'huile d'olive. Saler, poivrer et remettre
    au four 10 minutes environ.

  \item Au moment de servir, ajouter un filet de vinaigre balsamique
    et des copeaux de parmesan.

  \end{enumerate}

\end{recipe}

\stage{Astuces}
\begin{enumerate}

\item Sortir les rouleaux de p\^ate \`a l'avance - la p\^ate est alors
  plus facile \`a travailler.

\item Piquer la p\^ate \`a l'aide d'une fourchette l'empêche de
  gonfler \`a la cuisson

\item Faire pr\'ecuire les disques de p\^ate au four entre deux
  plaques pour qu'ils restent bien plats.

\item Utiliser des tomates roma ou olivette. Plus fermes
  elle ne d\'etremperont pas votre p\^ate

\item Couper les tomates plus facilement \`a l'aide d'un couteau scie
  (genre couteau \`a pain).

\item Les tartes sont cuites lorsque les tomates sont molles en
  surface et juteuses.

\item Pour faire de jolis copeaux de parmesan, utilisez un
  \'econome.

\item On peut aussi, rajouter un peu de basilic frais cisel\'e au
  moment de servir.

\end{enumerate}