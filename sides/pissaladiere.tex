\dish{Pissaladi\`ere aux oignons}
\prep{1 hour(?)}
\source{CVF-103,stephane.birkle}

\begin{ingredients}
  \ingr{350}{g}{p\^ate \`a pain}
  \ingr{1}{kg}{oignon}
  \ingr{150}{g}{raisins secs}
  \ingr{1}{cuil.}{gingembre frais, r\^ap\'e}
  \ingr{3}{cuil. \`a soupe}{vinaigre de X\'er\`es}
  \ingr{20}{g}{beurre}
  \ingr{1}{cuil. \`a soupe}{farine}
  \ingr{3}{cuil. \`a soupe}{huile d'olive}
  \ingr{}{}{sel}
  \ingr{}{}{poivre}
\end{ingredients}


\begin{recipe}
  \begin{enumerate}

  \item Faire gonfler les raisins.

  \item Chauffer le four.

  \item Peler et \'emincer les
    oignons, les faire fondre doucement 15 min avec l'huile d'olive dans
    une sauteuse.

  \item Ajouter alors le gingembre, la farine, m\'elanger puis
    mouiller avec le vinaigre, saler, poivrer, faire cuire encore 10 min.

  \item Ajouter les raisins.  \'Etaler la p\^ate et garnir un moule. Etaler la
    pr\'eparation sur la p\^ate.  Enfourner 25 min.

  \item A d\'eguster ti\`ede.

  \end{enumerate}

  \textbf{Jeff:} La vinaigre basalmique est plus douce et donne un
  gout plus agr\'eable.  Sinon, la recette originale propose une
  demi-tablette de bouillon de volaille, ce qui adoucerait sans doute
  la pissaladi\`ere.
\end{recipe}

