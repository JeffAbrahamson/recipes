\dish{Pita}
\altdish{Bread, pita}
%\serves{}
\makes{10}
\prep{1h over 25 h}
\source{modernist.bread}
\label{pita}

\begin{ingredients}
  \ingr{340}{ml}{water}
  \ingrS{13}{g}{instant dry yeast}
  \ingr{560}{g}{bread flour}
  \ingrS{25}{g}{sugar}
  \ingr{11}{g}{salt}
  \ingr{40}{g}{olive oil}
  \ingr{35}{g}{canola oil}
\end{ingredients}


\begin{recipe}
  \begin{enumerate}

  \item Mix water and yeast, then add flour and sugar.  Mix to shaggy
    mass.  Allow to rest briefly.

  \item Add salt, olive oil, and canola oil.  Mix on medium speed to
    full gluten development.

  \item Transfer to a lightly oiled bowl and cover well.  Bulk ferment
    1~hour with book fold at 30 minutes.

  \item Divide into ten 100~g pieces, preform as balls.  Oil lightly,
    cover with room to expand, and refrigerate 24~hours.

  \item Flatten balls and roll out to 18~cm.  Allow to sit 10~minutes
    to form a light crust, then bake 4\fracH--5~minutes at (ventilated)
    230\degreeC.  Pita is done when it starts to puff up and to
    develop slight brown tones.

  \end{enumerate}
\end{recipe}

Pita bread stales quickly, within a few hours.  Reheat just before
eating.  Freeze and reheat for longer storage.

%\accord{}
