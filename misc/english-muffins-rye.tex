\dish{Rye English Muffins}
\altdish{English Muffins, rye}
\altdish{Muffins, English, rye}
%\serves{}
\makes{12}
\prep{20 minutes + 20 minutes + 15--20 hours fermentation}
\source{kitchen-english-muffins}

\begin{ingredients}
  \ingr{40}{g}{levain}
  \ingr{90}{ml}{water}
  \ingrS{65--70}{g}{rye flour}
  \ingr{225}{g}{whole milk}
  \ingr{15}{g}{sugar}
  \ingr{7}{g}{salt}
  \ingr{25}{g}{butter}
  \ingr{200}{g}{rye flour}
  \ingrS{200--250}{g}{bread flour, T65}
  \ingr{}{}{cornmeal or semolina}
\end{ingredients}


\begin{recipe}
  \begin{enumerate}

  \item Mix levain, water, and 65~g flour.  Cover and let ferment
    8--12~hours.

  \item Mix in remaining ingredients.  Knead.  The dough will be
    tacky.  Set in an oiled bowl, cover, and let rise.

  \item Degas and divide the dough into 12~balls.  Flatten
    somewhat, dust with cornmeal, and let rise until puffy, 2--4~hours
    depending on temperature.
    
  \item Warm a skillet, add butter, and cook muffins six minutes per
    side.  The temperature should be warm enough that the muffins
    finish with a golden colour.  On my stove on rue Metzinger in
    Nantes, this is a medium-low heat.  Transfer to a wire rack.
    (Note that the butter is optional, and there's a risk that the
    butter burns during cooking.)

  \end{enumerate}
\end{recipe}

%\accord{}
