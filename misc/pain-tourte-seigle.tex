\dish{La tourte de seigle}
\altdish{Pain de seigle (tourte)}\
%\serves{}
\makes{500 g ou 1 kg}
\prep{36 heures}
\source{paul.magnette}

\begin{minipage}{.49\textwidth}
  \stage{pain de 500 g}

\begin{ingredients}
  \ingr{30}{g}{levain}
  \ingr{50}{g}{froment bise T80, T100, ou de seigle}
  \ingrS{30}{g}{eau tiède}
  \ingr{50}{g}{froment bise T80, T110, ou de seigle}
  \ingrS{30}{g}{eau tiède}
  \ingr{190}{g}{farine de seigle complète}
  \ingr{160}{g}{eau tiède}
  \ingr{5}{g}{sel fin}
\end{ingredients}
\end{minipage}
\begin{minipage}{.49\textwidth}
  \stage{pain de 1000 g}
  
\begin{ingredients}
  \ingr{60}{g}{levain}
  \ingr{100}{g}{froment bise T80, T100, ou de seigle}
  \ingrS{60}{g}{eau tiède}
  \ingr{100}{g}{froment bise T80, T100, ou de seigle}
  \ingrS{60}{g}{eau tiède}
  \ingr{380}{g}{farine de seigle complète}
  \ingr{320}{g}{eau tiède}
  \ingr{10}{g}{sel fin}
\end{ingredients}
\end{minipage}

\begin{recipe}
  \begin{enumerate}

  \item \textbf{L'avant veille}, prélever levain-mère bien fraîchi,
    ajouter farine et eau, mélanger vigoureusement à la cuillère.
    Couvrir et laisser reposer 8~h à température ambiante.
    
  \item \textbf{La veille}, nourrir une nouvelle fois le levain,
    mélanger, couvrir, et laisser reposer 4~h à température ambiante.
    
  \item Verser l'eau et le levain ensemble, diluer à l'aide d'une
    spatule.  Verser la farine de seigle, mélanger, saler, et laisser
    reposer 30~minutes.
    
  \item Travailler la pâte à la main quelques minutes, couvrir et
    laisser reposer 2~h à température ambiante.
    
  \item Déposer le pâton sur un plan de travail bien fariné, lui
    donner une forme ronde sans trop le travailler et en farinant les
    doigts autant que nécessaire pour qu'ils n'adhèrent pas à la
    pâte.  Placer le pâton dans un banneton bien fariné, le pli contre
    la toile, la partie lisse vers l'extérieur.  Laisser reposer 1h30
    à température ambiante.
    
  \item Pendant ce temps préchauffer le four à 250\degreeC{} et placer la
    pierre de cuisson à mi-hauteur, la lèchefrite étant dans le bas du
    four.
    
  \item Lorsque le four et la pierre sont bien chauds, déposer le
    pâton sans le grigner.  Jeter un verre d'eau dans la lèchefrite.
    
  \item Après 20~minutes, baisser la température à 200\degreeC, ouvrir le
    four brièvement pour laisser évacuer la vapeur, et laisser cuire
    40~minutes de plus.
    
  \item À la fin de la cuisson, laisser le pain dans le four éteint et
    entrouvert 15~minutes, puis le faire ressuer un jour entier sur
    une volette avant de le consommer.
    

  \end{enumerate}
\end{recipe}

%\accord{}
