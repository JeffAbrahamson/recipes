\dish{Pâte feuilletée}
%\altdish{}
%\serves{}
\makes{575 g}
\prep{1 heure sur 4 heures}
\source{vincent.guerlais}
\label{pâte-feuilletée}

\stage{Détrempe}

\begin{ingredients}
  \ingr{250}{g}{farine T55}
  \ingr{50}{g}{beurre}
  \ingr{5}{g}{sel}
  \ingrS{2,5}{g}{sucre (si usage sucré)}
  \ingr{125}{ml}{eau}
\end{ingredients}

\begin{recipe}
  \begin{enumerate}

  \item Mélanger farine, beurre, et sel.  Ajouter l'eau.
    Pétrir brièvement à la main, former un oval, filmer, et mettre au
    frigo pendant au moins 1~heure.
  \end{enumerate}
\end{recipe}
  
\stage{Tourage}

\begin{ingredients}
  \ingr{150}{g}{beurre}
\end{ingredients}

\begin{recipe}
  \begin{enumerate}
  \item Malaxer le beurre dans un torchon humide, former un oval,
    filmer, et mettre au frigo.
    
  \item Rouler la détrempe entre deux feuille de papier sulfarisé pour
    obtenir un rectangle approximativement 12$ \times $20~cm.  Rouler
    le beurre à $12\times 12$~cm.  Placer le beurre sur la détremple,
    plier un tiers de la détrempe sur le beurre, puis plier en deux
    pour créer deux couches de beurre et trois de détrempe.
    
  \item Rouler dans l'autre sens à $12\times 24$~cm, plier les 6~cm de
    chaque bout vers le milieu, puis filmer et remettre au moins une
    heure au frigo.  (Elle peut y rester la nuit.)  C'est deux tours.
    
  \item Faire encore deux tours : rouler à $12\times 18$~cm et plier
    en tiers.  Filmer et remettre au frigo au moins une heure.  (Elle
    peut y rester la nuit.)
    
  \item Rouler et former le plat final.

  \end{enumerate}
\end{recipe}

%\accord{}
