\dish{Benoîtons}
%\altdish{}
%\serves{}
\fait{12-18 benoîtons}
\prep{20 minutes + 3,5 heures}
\source{encycl.pain}

\begin{ingredients}
  \ingr{350}{g}{eau}
  \ingrS{5}{g}{levure sèche}
  \ingr{100}{g}{levain}
  \ingr{300}{g}{farine T150}
  \ingr{200}{g}{farine de seigle}
  \ingr{7}{g}{canelle moulue}
  \ingrS{9}{g}{sel}
  \ingrS{30}{g}{beurre mou}
  \ingr{300}{g}{raisins secs de Corinthe}
\end{ingredients}


\begin{recipe}
  \begin{enumerate}

  \item Mettre tous les ingrédients sauf le beurre et les raisins secs
    dans le bol d'un robot.  Pétrir 3~minutes à vitesse~1, puis
    ajouter le beure en morceau et pétrir encore 4--6~minutes à
    vitesse~2.  Tout à la fin ajouter des raisins secs.
    
  \item Laisser la pâte 2~heures couverte jusqu'à ce qu'elle double en
    volume, en faisant un rabat toutes les 30~minutes.
    
  \item Étaler la pâte au rouleau sur un plan de travail fariné sans
    appuyer pour former un rectangle de 2~cm d'épaisseur.
    
  \item Détailler en bâtons de 3~cm de largeur.  Les placer sur une
    plaque.
    
  \item Couvrir et laisser couvert 1~heure.
    
  \item Préchauffer le four à 200\degreeC.

  \item Enfourner le plaque, verser un verre d'eau sur la lèchefrite,
    et laisser cuire 15~minutes.
    
  \end{enumerate}
\end{recipe}

\notes Les variations sucrées (chocolat, \dots) ainsi que salées
(oignons, fêta, \dots) sont envisageables.

%\accord{}
