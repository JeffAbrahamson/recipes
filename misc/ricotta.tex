\dish{Ricotta}
%\altdish{}
%\serves{}
\fait{350 g de ricotta}
\prep{15~minutes + 1~heure}
\source{internet}

\begin{ingredients}
  \ingr{1}{litres}{lait}
  \ingr{\fracH}{litre}{crème liquide entière}
  \ingrS{3,3}{g}{sel}
  \ingr{60}{ml}{acide citrique (jus d'un citron, par exemple)}
\end{ingredients}


\begin{recipe}
  \begin{enumerate}

  \item Mélanger le lait, crème, et sel dans une casserole.  Chauffer
    doucement à 90\degreeC, remuant occasionnellement afin d'éviter la
    formation d'une peau.
    
  \item Ajouter l'acide citrique.  Remuer, puis laisser cailler
    jusqu'à la séparation du petit lait.
    
  \item Égoutter à travers une mousseline jusqu'à l'obtention de la
    consistance de ricotta souhaitée (typiquement entre 15--60~minutes).

  \end{enumerate}
\end{recipe}

Notes :
\begin{itemize}
\item Le vinaigre blanc peut substituer à l'acide citrique.
\item Plus d'acide fait des granulés de lait plus gros.
\item La température n'est pas un seuil : le lait caille mieux avec la
  chaleur et on ne souhaite pas, ici, le faire bouillir.
\end{itemize}


%\accord{}
