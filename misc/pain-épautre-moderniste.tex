\dish{Pain de grand épautre}
\altdish{Bread, spelt}
\altdish{Bread, einkorn}
\altdish{Spelt bread}
\altdish{Einkorn bread}
\altdish{Grand épautre, pain de}
\altdish{Petite épautre, pain de}
\altdish{Pain de petite épautre}
%\serves{}
\makes{600 g}
\prep{six heures}
\source{jma}

\begin{ingredients}
  \ingr{1}{g}{levure de boulanger}
  \ingr{100}{ml}{eau tiède}
  \ingr{225}{g}{levain de grand épautre}
  \ingr{330}{g}{farine de grand épautre}
  \ingr{6--7}{g}{sel}
\end{ingredients}


\begin{recipe}
  \begin{enumerate}

  \item Mélanger l'eau tiède et la levure de boulanger.  Ajouter les
    autre ingrédients, puis mélanger et pétrir à la machine.  Ajouter
    de la farine pour avoir une bonne pâte.
    
  \item Laisser reposer 2$\,$\fracH heures.  Plier en deux quatre fois toutes
    les heures (donc deux fois).
    
  \item Pré-former la pâte.  Laisser reposer 15--20 minutes.
    
  \item Former le pain.  Laisser reposer à température ambiente 1--3
    heures selon la température ambiente.
    
  \item Mettre dans un four préchauffé à 240\degreeC.  Baisser la
    température à 160\degreeC après chargement du four.  Laisser cuir
    35~minutes.  Laisser refroidir sur une volette.

  \end{enumerate}
\end{recipe}

\stage{Pain de petite épautre}

Faire le levain toujours avec grand épautre.

Au lieu d'ajouter 330~g de grande épautre, ajouter 300~g de farine de
petite épautre ainsi que 30~g de gluten.

%\accord{}
