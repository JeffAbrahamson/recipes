\dish{Pain au lait}
\altdish{Lait, pain au}
%\serves{}
\makes{8}
\prep{3 heures}
\source{encycl.viennoiserie}

\begin{ingredients}
  \ingr{7}{g}{levure sèche}
  \ingrS{200}{g}{lait entier, tiède}
  \ingr{500}{g}{farine, T65}
  \ingr{10}{g}{sel}
  \ingr{70}{g}{sucre}
  \ingrS{2}{}{oeufs}
  \ingrS{125}{g}{beurre ramolli, coupé en morceaux}
  \ingr{1}{}{jaune d'oeuf}
  \ingr{8}{\cs}{sucre perlé}
\end{ingredients}


\begin{recipe}
  \begin{enumerate}

  \item Délayer la levure dans le lait tiédi.

  \item Ajouter la farine, le sel, le sucre et les oeufs.  Pétrir
    10~minutes à vitesse~1.

  \item Ajouter le beurre coupé en morceaux, puis pétrir 5~minutes à
    vitesse~2.

  \item Laisser fermenter 2~heures, couvert.

  \item Diviser en~8, pré-former en boules, puis former des petits
    pains longs.  Les espacer sur une plaque couverte de papier
    cuisson.  Couvrir et laisser fermenter 45~minutes.

  \item Préchauffer le four à 200\degreeC{}, chaleur tournante.

  \item Faire une grigne sur le dessus dans le sens long.  Dorer avec
    le jaune d'oeuf et parsemer de sucre perlé.

  \item Enfourner 10~minutes jusqu'à ce qu'ils soient dorés.

  \item Laisser refroidir sur une volette avant de les consommer tiède.

  \end{enumerate}
\end{recipe}

%\accord{}
