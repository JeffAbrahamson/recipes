\dish{Falafel}
\altdish{}
%\serves{}
\makes{40 falafel balls}
\prep{1 hour}
\source{internet}

\stage{Falafel}

\begin{ingredients}
  \ingr{400}{g}{dry chickpeas, soaked overnight}
  \ingr{1--2}{}{onions, chopped}
  \ingr{4-6}{cloves}{garlic, chopped}
  \ingr{2}{\tsp}{salt}
  \ingr{\fracH}{\tsp}{ground black pepper}
  \ingr{1}{\tbsp}{coriander powder}
  \ingr{2}{\tsp}{cumin powder}
  \ingr{1}{\tsp}{cardamom}
  \ingr{1}{c}{coriander leaves, chopped}
  \ingr{1}{c}{parsley leaves, chopped}
\end{ingredients}

\begin{recipe}
  \begin{enumerate}

  \item Chop the chickpeas in a food processor.  Then add remaining
    ingredients and chop further until the texture is of course sand.

  \item Allow to rest 30--60~minutes in the refrigerator.

  \item Form balls, slightly oblate for uniform cooking.

  \item Cook 2--4~minutes (until golden but not charred) in oil at
    175--180\degreeC, turning once mid-way through cooking.

  \end{enumerate}
\end{recipe}

One can also use fava beans or lentils.  The recipe is ancient and the
world has seen many variations.

\vbox{
\stage{Tahini}

\begin{ingredients}
  \ingr{160}{g}{sesame paste}
  \ingr{2}{clove}{garlic}
  \ingr{4}{\tsp}{lemon juice}
  \ingr{1}{\tbsp}{salt}
  \ingr{\fracQQ}{c}{water}
  \ingr{1}{\tbsp}{sugar (optional)}
  \ingr{2}{\tbsp}{minced parsley (optional)}
\end{ingredients}

\begin{recipe}
  \begin{enumerate}

  \item Mix together.

  \end{enumerate}
\end{recipe}
}

\vbox{
\stage{Yogurt Sauce}

\begin{ingredients}
  \ingr{250}{g}{Greek yogurt}
  \ingr{1}{}{shallot}
  \ingr{1}{clove}{garlic}
  \ingr{\fracH}{\tsp}{salt}
  \ingr{\fracQ}{\ccf}{ground cumin}
  \ingr{1}{\cs}{lemon juice}
  \ingr{5}{g}{chive, chopped}
\end{ingredients}


\begin{recipe}
  \begin{enumerate}

  \item Mix.

  \end{enumerate}
\end{recipe}

\accord{Eat with fresh \hyperref[pita]{pita}.  Serve with chopped
  onion, tomato, etc.}
}
