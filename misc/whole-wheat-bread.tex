\dish{Whole Wheat Bread}
\altdish{Bread, whole wheat}
\makes{2 loaves}
\prep{1 hour plus 2 hours for baking and cooling}
\source{jma}

\begin{ingredients}
  \ingr{2}{c}{warm water}
  \ingr{2}{Tbsp}{dried yeast}
  \ingr{6}{c}{whole wheat flour}
  \ingr{2}{Tbsp}{honey}
  \ingr{1}{Tbsp}{salt}
\end{ingredients}


I make bread using a KitchenAid mixer, which permits me to add all of
the flour at once, adjusting at the end if I didn't add enough.  (I
can add water at the end if I added too much, but the result is not
quite as good.)

If kneading by hand, add only half of the flour, then slowly add more
as you knead it.


\begin{recipe}
  \begin{enumerate}

  \item In mixing bowl, gently pour yeast on top of warm water that is
    just barely hot to the touch.  Let sit several minutes without
    stirring for yeast to dissolve.

  \item Add flour (precise quantity depends heavily on humidity and
    any other items you add), honey, and salt.  Mix until thick and
    then knead well.

  \item Let rise until a finger poked into the bread easily makes a
    whole.  Punch down, form loaves, dust well with whole wheat flour,
    and place on a cookie sheet dusted with whole wheat flour.
    Preheat oven to 425\F.

  \item When loaves have risen (doubled), place in oven.  Reduce
    temperature to 350\F\ and bake about 40 minutes, until loaves, when
    tapped on the bottom, sound hollow.

  \item Cool thoroughly on wire racks.

  \end{enumerate}

  Note that almost anything can be added.  Oil, butter, or grated
  cheese makes for a richer loaf.  Spices or finely chopped (and
  cooked) vegetables work well, too.

\end{recipe}
