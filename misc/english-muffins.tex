\dish{English Muffins}
\altdish{Muffins, English}
%\serves{}
\makes{12}
\prep{20 minutes + 20 minutes + 2--20 hours fermentation}
\source{kitchen-english-muffins}

\begin{ingredients}
  \ingr{110}{ml}{water (orig: \fracH c)}
  \ingr{1}{g}{dried yeast (orig: \fracH \tsp)}
  \ingrS{85--90}{g}{spelt flour (orig: \fracQQ c bread flour)}
  \ingr{225}{g}{whole milk (orig: 1 c)}
  \ingr{20--25}{g}{sugar (orig: 2 \tbsp = 30 g)}
  \ingr{7}{g}{salt (orig: 1 \tsp)}
  \ingr{25}{g}{butter (orig: 2 \tbsp)}
  \ingrS{400--500}{g}{spelt flour (orig: 3--3\fracQ c bread flour)}
  \ingr{}{}{cornmeal or semolina}
\end{ingredients}


\begin{recipe}
  \begin{enumerate}

  \item Mix Water, yeast, and 85~g flour.  Cover and let ferment
    1--12~hours (depending on ambient fermentation temperature) until
    bubbly.

  \item Mix in remaining ingredients.  Knead.  The dough will be
    tacky.  With an electric mixer, one tends to add more flour to
    avoid a too-sticky dough.  Set in an oiled bowl, cover, and let
    rise.

  \item Punch down.  Divide the dough into 12~balls.  Flatten
    somewhat, dust with cornmeal, and let rise until puffy, 1--2~hours
    depending on temperature.
    
  \item Warm a skillet, add butter, and cook muffins six minutes per
    side.  The temperature should be warm enough that the muffins
    finish with a golden colour.  On my stove on rue Metzinger in
    Nantes, this is a medium-low heat.  Transfer to a wire rack.

  \end{enumerate}
\end{recipe}

Notes:
\begin{itemize}
\item The original recipe called for adding 1~\tsp of dried yeast to
  the milk.  This seems not to matter.
\item To make crumpets, add \fracQQ~c extra milk to form a very loose
  batter.  Use rings to shape in the pan.  Drop room temperature dough
  onto the griddle and cook until bubbly (like pancakes).
\end{itemize}


%\accord{}
