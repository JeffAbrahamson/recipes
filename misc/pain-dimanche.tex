\dish{Mon Pain du Dimanche}
\altdish{Dimanche, pain du}
\altdish{Pain du Dimanche}
%\serves{}
\fait{2 kg}
\prep{40 heures}
\source{paul.magnette}

\begin{ingredients}
  \ingr{60}{g}{levain}
  \ingr{120}{g}{eau}
  \ingrS{180}{g}{farine de blé T150}
  \ingr{100}{g}{eau}
  \ingr{}{pincée}{levure sèche}
  \ingrS{100}{g}{farine de blé T65}
  \ingr{600}{g}{eau}
  \ingr{100}{g}{farine de seigle T130}
  \ingr{150}{g}{farine de blé T150}
  \ingr{550}{g}{farine de blé T65}
  \ingr{18}{g}{sel}
\end{ingredients}



\begin{recipe}
  \begin{enumerate}

  \item \textbf{Vendredi soir : } Prélever 60~g de levain-mère, ajouter
    eau et farine, mélanger.  Couvrir et laisser reposer la nuit.
    
  \item Mélanger 100 g d'eau, la pincée de levure sèche, et 100~g de
    farine pour faire la poolish.  Couvrir et laisser reposer la nuit.
    
  \item \textbf{Samedi matin : } Mélanger 600~g d'eau avec le levain
    épais.  Ajouter farine restante.  Mélanger juste pour que l'eau
    soit absorbée, puis laisser reposer, couvert, 30~minutes.
    
  \item Ajouter la poolish, mélanger très brièvement, puis laisser
    reposer, couvert, 30~minutes.
    
  \item Pétrir, ajoutant le sél mi-pétrissage.  Laisser reposer,
    couvert, 30~minutes.
    
  \item Plier la pâte sur elle-même.  Laisser reposer, couvert,
    30~minutes.
    
  \item Plier encore une fois la pâte sur elle-même.  Laisser reposer,
    couvert, 3~heures.
    
  \item Fariner légèrement le plan de travail, sortir le pâton,
    replier la pâte sur elle-même sans la travailler pour conserver le
    plus de gaz carbonique possible.  Filmer et laisser reposer
    15~minutes.
    
  \item Façonner le pâton en le travaillant le plus délicatement
    possible en une boule légèrement ovale (afin, selon son banneton).
    Déposer en banneton.  Couvrir et placer dans le bas du frigo
    jusqu'au lendemain.
    
  \item \textbf{Dimanche matin : } Sortir la pâte du frigo.  Préchauffer le
    four à 250\degreeC.

  \item Lorsque le four est chaud, fleurer une plaque de farine de blé
    T150.  Déposer la pâte sur la plaque et enfourner.
    
  \item Après 20~minutes, réduire la chaleur à 200\degreeC.  Prolonger
    la cuisson 40~minutes.
    
  \item Éteindre et ouvrir le four.  Laisser le pain 10~minutes dans
    le four avec la porte entreouverte.
    
  \item Transférer le pain sur une volette et laisser ressuer au moins
    2~heures avant de le déguster.

  \end{enumerate}
\end{recipe}

%\accord{}
