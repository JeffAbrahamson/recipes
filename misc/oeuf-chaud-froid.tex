\dish{Oeuf Chaud Froid}
%\altdish{}
%\serves{}
\makes{1}
\prep{10 minutes}
\source{alain.passard}

\begin{ingredients}
  \ingr{1}{}{oeuf frais}
  \ingr{1}{}{pinc\'ee de quatre \'epices}
  \ingr{\fracH}{\tsp}{vinaigre de X\'er\`es}
  \ingr{}{}{poivre noir}
  \ingr{2}{}{pinc\'ees de fleur de sel (s\'epar\'ement)}
  \ingr{1}{\tbsp}{cr\`eme fleurette}
  \ingr{\fracH}{\tsp}{sirop d'\'erable}
\end{ingredients}


\begin{recipe}
  \begin{enumerate}

  \item Couper la coquille d'oeuf avec un tocqueur d'oeuf.  Evacuer le
    blanc et r\'eserver le jaune au fond de la coquille.  Recouvrir
    d'une pinc\'ee de fleur de sel et d'une pinc\'ee de poivre
    fra\^ichement concass\'e.  R\'eserver l'oeuf dans son alv\'eole.

  \item Fouetter la cr\`eme fleurette avec le vinaigre, le quatre
    \'epices, la fleur de sel, et un tour de moulin de poivre.

  \item Sur une eau fr\'emissante (60\degreeC), faire flotter 10~minutes la
    coquille d'oeuf avec son jaune.  Ensuite ajouter dans la coquuille
    la cr\`eme pr\'ealablement fouett\'ee et le sirop d'\'erable.

  \item Servir en coquette.

  \end{enumerate}
\end{recipe}
