\dish{Poivrons grillés marinés au thym citron et à l'ail rose}
%\altdish{}
\serves{6}
%\makes{}
\prep{15 minutes (plus 30m cuisson et 2h réfrigération)}
\source{CVF-hs-ete}

\begin{ingredients}
  \ingr{3}{}{poivrons rouges}
  \ingr{1}{}{tête de l'ail rose}
  \ingr{1}{}{citron (jus de)}
  \ingr{6}{\cs}{huile d'olive}
  \ingr{6}{brins}{thym citron}
  \ingr{}{}{sel}
  \ingr{}{}{poivre}
\end{ingredients}


\begin{recipe}
  \begin{enumerate}

  \item Allumer le four en position gril.  Laver et sécher les
    poivrons.  Les poser sur la gille du four.  La glisser à
    mi-hauteur du four et installer la lèchefrite en dessous.  Faire
    griller les poivrons en les retournant régulièrement jusqu'à ce
    qu'ils soient uniformement bruns (environ 30~minutes).

  \item Sortir les poivrons du four et les enfermer dans un sac en
    plastique.  (Ils seront plus faciles à peler).  Laisser refroider.

  \item Peler les poivrons et les couper en deux.  Retirer les graines
    et les filaments blancs.  Détailler la chair en lanières.   Les
    mettre dans un saladier.  Ajouter les gousses d'ail pelées et
    grossièrement émincées ainsi que le jus de citron, l'huile
    d'olive, et le thym.  Saler, poivrer, et mélanger.  Couvrer le
    plat d'un film étirable et réserver au frais pendant deux heures
    minimum.

  \item Sortir les poivrons marinés 30~minutes avant de les servir.
    Déguster avec des tranches de pain de campagne grillé.

  \end{enumerate}
\end{recipe}

\accord{un bordeaux clairet}
