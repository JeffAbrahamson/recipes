\dish{Beurre blanc au curry}
%\altdish{}
\serves{4}
%\makes{}
\prep{35 minutes}
\source{unknown}

\begin{ingredients}
  \ingr{3}{}{\'echalotes (80 g)}
  \ingr{100}{ml}{muscadet}
  \ingr{50}{ml}{eau}
  \ingrS{1}{\cs}{vinaigre de riz}
  \ingr{8}{g}{curry de madras}
  \ingrS{200}{g}{beurre}
  \ingr{}{}{sel}
  \ingr{}{}{poivre}
\end{ingredients}


\begin{recipe}
  \begin{enumerate}

  \item Émincer les échalotes.  Réduire presqu'à sec avec le muscadet,
    eau et vinaigre sur feu moyen.

  \item Ajouter le curry de madras puis le beurre froid peu à peu en
    fouettant sur feu doux.  La préparation doit être légèrement
    crémeuse (blanche).

  \item Ajouter du sel ou du poivre si souhaité.  Servir chaud (ou
    garder en bain marie sinon).

  \end{enumerate}
\end{recipe}

\bigskip Faire cuire au four 4~pommes de terre de 300~g chacune.
Couper en deux et servir couvert du beurre blanc.

\bigskip
\textit{Source : Par Nhung Phung, Song Saveurs et Sens (Nantes)}

