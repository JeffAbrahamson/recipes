\dish{Gnocchi}
\altdish{Potato Gnocchi}
\serves{4 people}
%\makes{}
\prep{30 minutes plus an hour to bake potatoes and time to make sauce}
\sourceD{nyt}{Mark Bittman, 21 February 2013}


\begin{ingredients}
  \ingr{700}{g}{starchy potatoes (e.g., charlotte, russet)}
  \ingr{}{}{salt}
  \ingr{}{}{pepper}
  \ingr{\fracH--\,\fracQQ}{c}{flour}
\end{ingredients}


\begin{recipe}
  \begin{enumerate}

  \item Heat oven to \CF{200}{400}.  Bake potatoes until tender, about
    an hour.  Immediately split them open to let the steam
    escape. When you can handle the potatoes, scoop out their flesh.

  \item Bring a large pot of water to a boil and salt it. Pass potato
    flesh through a ricer or food mill, and season to taste. Sprinkle
    \fracH cup flour on a clean counter or cutting board, and knead the
    potatoes with it, sprinkling in the remaining \fracQ cup flour, until
    the dough just comes together. Pinch off a piece of the dough, and
    boil it to make sure it will hold its shape. If it does not, knead
    in a bit more flour (no more than necessary), and try again; the
    gnocchi will float to the top and look a little ragged when ready.

  \item Roll a piece of the dough into a rope about 1/2-inch thick,
    then cut the rope into 1/2-inch lengths. Score each piece by
    rolling it along the tines of a fork; as each piece is ready, put
    it on a baking sheet lined with parchment or wax paper; do not
    allow the gnocchi to touch one another.

  \item Add gnocchi to the boiling water a few at a time, and stir
    gently; adjust the heat so the mixture doesn’t boil too
    vigorously. A few seconds after they rise to the surface, the
    gnocchi are done; remove them with a slotted spoon or mesh
    strainer, and finish with any of the sauces below.

  \end{enumerate}
\end{recipe}

\stage{Flavors}

\textbf{Beet Gnocchi:} Peel and grate 1/2 pound beets. Cook in 2
tablespoons olive oil over medium-low heat, seasoning to taste, until
very soft, 25 to 30 minutes. Transfer to food processor, and purée
until smooth. Stir into the mashed potatoes in Step 2 of the master
recipe (you’ll most likely need an extra 1/4 cup flour).

\textbf{Spinach Gnocchi:} Roughly chop 8 ounces spinach. Cook in 2
tablespoons olive oil over medium-low heat, seasoning to taste and
stirring, until soft and wilted, about 5 minutes. Rinse under cold
water and squeeze dry. Transfer to food processor and purée until
smooth. Stir into the mashed potatoes in Step 2 of the master recipe
(you’ll most likely need an extra 1/4 cup flour).

\textbf{Carrot Gnocchi:} Peel and grate 1/2 pound carrots. Cook in 2
tablespoons olive oil over medium-low heat, seasoning to taste, until
very soft, 20 to 30 minutes. Transfer to food processor and purée
until smooth. Stir into the mashed potatoes in Step 2 of the master
recipe (you’ll most likely need an extra 1/4 cup flour).

\stage{Sauces}

\textbf{Tomato Sauce:} Cook a small chopped onion in olive oil until
soft. Add minced garlic, 3 to 4 cups chopped tomatoes, canned or
fresh, and salt and pepper. Cook at a steady bubble until "saucy." If
the sauce becomes too thick, add a splash of the gnocchi cooking water
before serving. Garnish with torn basil and/or grated Parmesan.

\textbf{Brown Butter, Sage and Parmesan:} Put 4 tablespoons butter and
a handful of fresh sage leaves (40 wouldn’t be too many) in a skillet
over medium heat. Cook until the butter is light brown and the sage is
sizzling, about 3 minutes. Toss with the gnocchi, some of their
cooking water and loads of grated Parmesan.

\stage{Tips}

A phrase often used (overused, really) to describe well-made gnocchi
is "light as a cloud." It’s not an especially instructive description
for a piece of real food, and for cooks hoping to try their hands at
gnocchi for the first time, it can be downright daunting.

It’s true that gnocchi requires a bit of technique, but achieving that
cloudlike texture — "light" is perhaps a simpler, less intimidating
word — isn’t actually that difficult.

It’s all in the dough. There are just a few keys to remember: 1) Use
starchy potatoes, like regular old russets (baking potatoes); nothing
fancy. It’s the starch from the potatoes — along with the gluten from
the flour — that holds the dough together. 2) You don’t want
overcooked, waterlogged potatoes; the wetter they are, the more flour
you’ll need. I bake them whole, which is effortless, but you could
also boil them whole and unpeeled. If time allows, you might dry them
out in a low oven for a little while, once they’re fully tender. 3)
Use as little flour as you can get away with to make the dough hold
its shape. Add the flour a little at a time, and test-boil a piece of
dough — even if you think it’s not ready yet — to see if it holds
together. 4) Be gentle when mixing and kneading; the idea is to avoid
overdeveloping the gluten — which is the offense most likely to make
your gnocchi decidedly un-cloudlike.

Another option is to add an egg, which makes success more likely but —
and reasonable people disagree about this — makes the final product a
tad heavier.

Once you’ve got the basic recipe down, you can start messing around
with other vegetables in combination with potatoes. Carrots, beets and
spinach are all terrific — especially for their colors — as are
squash, parsnips, sweet potatoes, chard and kale. Cook them in olive
oil until soft, purée them in the food processor and mix them into the
cooked potatoes — because the vegetables carry some extra moisture,
you’ll most likely need a little more flour.

Whatever version you make, all it needs is a simple sauce — and not
too much of it. The gnocchi is the star — or should I say, the cloud?
