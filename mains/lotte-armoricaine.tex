\dish{Lotte à l’armoricaine}
%\altdish{}
\pour{4}
%\makes{}
\prep{1h}
\source{cvf-web}

\begin{ingredients}
  \ingr{2}{}{oignons}
  \ingr{1}{gousse}{ail}
  \ingrS{3}{\cs}{huile d’olive}
  \ingr{800}{g}{lotte coupée en médaillons épais}
  \ingr{50}{g}{farine}
  \ingrS{$2\times 5$}{cl}{cognac}
  \ingrS{1}{\cs}{huile d’olive}
  \ingr{800}{g}{pulpe de tomates en boîte, égoutées}
  \ingr{2}{\cs}{concentré de tomates}
  \ingr{1}{morceau}{écorce d’orange séchée}
  \ingr{15}{cl}{vin blanc sec}
  \ingr{}{}{noix muscade}
  \ingr{}{}{Tabasco}
  \ingr{}{}{sel}
  \ingrS{}{}{poivre}
  \ingrS{75}{g}{sucre}
  \ingr{1}{bouquet}{persil plat}
\end{ingredients}


\begin{recipe}
  \begin{enumerate}

  \item Peler et émincer les oignons et l’ail.

  \item Fariner la lotte. Dans une sauteuse, faire dorer avec
    3~cuillerées d’huile.

  \item Arroser de 5~cl de cognac chauffé et faire flamber. Égoutter.

  \item A la place de la lotte, faire dorer les oignons et l’ail avec
    le reste d’huile.

  \item Ajouter les tomates égouttées, le concentré, le vin et le
    reste de cognac. Saler, poivrer, parfumer de muscade et relever
    d’un trait de Tabasco. Remuer 5 min sur feu moyen.
    
  \item Dans une petite casserole, faire un caramel blond avec le
    sucre. L'ajouter dans la sauteuse avec l’écorce d’orange et la
    lotte.  Remuer et laisser 15~minutes sur feu doux.

  \item Parsemer de persil ciselé et servir très chaud avec un mélange
    de riz blanc et riz sauvage.

  \end{enumerate}
\end{recipe}

\stage{L'écorce d'orange séchée}

Pour préparer les écorces séchées, laver et essuyer des oranges bio,
prélever le zeste en ruban sans la peau blanche. Dessécher à 120\degreeC,
porte mi-ouverte, au moins 2 h. Garder les écorces refroidies dans une
boîte hermétique.
