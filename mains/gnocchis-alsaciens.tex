\dish{Gnocchi Alsaciens de semoule}
\altdish{Alsatian Gnocchi}
%\serves{}
%\makes{}
%\prep{}
\source{theo.birkle}

\begin{ingredients}
  \ingr{1}{L}{lait}
  \ingr{200}{g}{semoule}
  \ingr{100}{g}{beurre}
  \ingr{1}{\ccf}{sel fin}
  \ingr{}{}{muscade rap\'ee}
  \ingr{2}{}{jaunes d'oeufs}
\end{ingredients}


\begin{recipe}
  \begin{enumerate}

  \item Faire cuire le lait avec la moitie du beurre et le sel.

  \item Ajouter la semoule en pluie fine, remuer jusqu'a obtenir une
    bouillie epaisse.

  \item Hors du feu ajuouter les oeufs et la muscade.

  \item Etaler cette pr\'eeparation sur une planche en bois ou une
    tourti\`ere et laisser refroidir et reposer 1~h.

  \item D\'ecouper ensuite cette masse au couteau ou \`a
    l'importe-pi\`ece en lonsanges ou rectangles.  Faire dorer des
    deux cot'es dans le beurre restant.

  \end{enumerate}

  Servir avec une sauce tomate, par exemple.
\end{recipe}
