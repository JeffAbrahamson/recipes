\dish{冷やし担々麺}
\aussidish{Hiyashi tantan-men}
\serves{4}
%\makes{}
\prep{1 hour (plus advance preparation)}
\source{cjm}


\stage{ragout}

\begin{ingredients}
  \ingrS{50}{g}{protéines de soja réhydratées (en petits morceaux, poids après réhydratation)}
  \ingr{1}{}{oignon}
  \ingr{2}{cm}{gingembre}
  \ingr{1}{gousse}{d'ail}
  \ingr{}{}{shitakés séchés du dashi shojin}
  \ingr{100}{g}{champignons}
  \ingrS{2}{\cs}{huile}
  \ingr{150}{ml}{\hyperref[精進だし]{精進だし}}
  \ingr{1,0}{\cs}{miso blanc}
  \ingr{1,0--1,5}{\cs}{sauce de soja}
  \ingr{1}{\cs}{sucre}
  \ingrS{1}{\cs}{mirin}
  \ingr{20}{g}{cacahouètes écrasées}
  \ingr{1}{\ccf}{huile de sésame}
\end{ingredients}


\begin{recipe}
  \begin{enumerate}

  \item Réhydrate les protéines de soja (tremper approx 30~minutes dans de l'eau chaude).

  \item Couper finement l'oignon, le gingembre, et l'ail.

  \item Retirer les pieds des shitakés, hacher les shitakés et les autres champignons.

  \item Dans un poêle faire chauffer l'huile, puis ajouter (avant que
    l'huile soit chaud) le gingembre et l'ail hachés.

  \item Ajouter l'oignon.

  \item Ajouter les champignons.

  \item Ajouter les protéines de soja, le dashi, le miso, la sauce de
    soja, le sucre, et le mirin.  Remuer jusqu'à ce que le liquide
    soit presque évaporé.

  \item Ajouter les cacahouètes et l'huile de sésame.

  \end{enumerate}
\end{recipe}


\stage{soupe}

\begin{ingredients}
  \ingrS{2}{grandes poignées}{feuilles d'épinards}
  \ingr{2}{\cs}{sauce de soja}
  \ingr{4}{\cs}{miso blanc}
  \ingrS{4}{\cs}{pâte de sésame}
  \ingr{400}{ml}{lait de soja}
  \ingrS{250}{ml}{dashi}
  \ingr{\fracH}{\ccf}{sel (au goût)}
\end{ingredients}


\begin{recipe}
  \begin{enumerate}

  \item Faire blanchir les feuilles d'épinard 1~minute.  Égouter,
    laisser tiédir, puis presser doucement pour enlever l'excédent
    d'eau.

  \item Dans un saladier diluer dans un peu de dashi la sauce de soja,
    le miso blanc, et la pâte de sésame.

  \item Saler si besoin.

  \end{enumerate}
\end{recipe}


\stage{composition}

\begin{ingredients}
  \ingr{}{}{\hyperref[拉麺]{拉麺}}
  \ingr{}{}{feuilles de coriandre}
  \ingr{}{}{grains de sésame}
  \ingr{}{}{râ-yu}
\end{ingredients}


\begin{recipe}
  \begin{enumerate}

  \item Repartir la soupe dans 4~bols.

  \item Faire cuire les nouilles ramen dans de l'eau chaude.  Égoutter
    et poser dans les bols de soupe.

  \item Répartir le ragoût dans les bols, puis les feuilles d'épinard.

  \item Ajouter les feuilles de coriandre, quelques grains de sésame.
    Arroser de râ-yu.

  \item Servir tiède.

  \end{enumerate}
\end{recipe}

