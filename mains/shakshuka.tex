\dish{Shakshuka}
%\altdish{}
\serves{4}
%\makes{}
\prep{30 minutes}
\source{internet}

\begin{ingredients}
  \ingr{1}{}{oignon, finement coupé}
  \ingr{2}{gousses}{ail, écrasé}
  \ingrS{}{}{huile d'olive}
  \ingr{1}{}{poivron rouge, coupé en dés}
  \ingrS{800}{ml}{tomate entière en boîte}
  \ingr{}{}{paprika}
  \ingr{}{}{cumin}
  \ingr{}{}{piment de cayenne}
  \ingr{}{}{sel}
  \ingr{}{}{poivre}
  \ingrS{}{}{éventuellement harissa}
  \ingrS{4}{}{oeufs}
  \ingr{}{}{coriandre, pain}
\end{ingredients}


\begin{recipe}
  \begin{enumerate}

  \item Faire revenir les oignon avec une cuillère de l'huile d'olive
    dans une poêle en fonte.  Ajouter l'ail écrasé.

  \item Ajouter le poivron.

  \item Ajouter les tomates, puis les épices au goût.

  \item Lorsque l'ensemble est chaud, ajouter les oeufs entiers et
    laisser mijoter une minute pour que les oeufs pochent.

  \item Servir aussitôt sur du pain avec du coriandre ciselé.

  \end{enumerate}
\end{recipe}

%\accord{}

Cette recette varie énormément de personne à personne.  C'est à faire selon son gré du moment.

Je mets le même nombre d'oeufs que le nombre de personnes que je
compte servir.  Seule, je pose les oeufs directement sur les tomates
dans la poêle à la fin.  En context plus formelle, je ferais des oeufs
pochés.
