\dish{Sicilian Pizza Dough}
\altdish{Pizza dough, Sicilian}
\serves{8}
%\makes{}
\prep{20 minutes + 7 hours}
\source{modernist.bread}

\begin{ingredients}
  \textbf{Poolish}\\
  \ingr{50}{g}{water}
  \ingr{50}{g}{bread flour}
  \ingrS{0.05}{g}{dry yeast}
  \textbf{Dough}\\
  \ingr{330}{g}{water}
  \ingrS{3}{g}{dry yeast}
  \ingrS{535}{g}{bread flour}
  \ingr{12}{g}{salt}
  \ingrS{5}{g}{olive oil}
  \ingr{}{}{olive oil for pan}
\end{ingredients}


\begin{recipe}
  \begin{enumerate}

  \item \textbf{Poolish:} Combine flour and water with a pinch of
    yeast (first three ingredients), stir well and ferment 12~hours.
    
  \item \textbf{Dough:} Combine water and yeast.  Add poolish.  Add
    flour.  Mix to shaggy mass.  Allow to autolyse, 30~minutes.
    
  \item Add salt and olive oil.  Mix to medium gluten development.
    Cover and allow to ferment 2\fracH~hours, with 1~book fold after
    each hour.  Check for full gluten development.  If the dough
    gasses too quickly, put it in the refrigerator for bulk fermentation.
    
  \item Drizzle a baking pan copiously with oliver oil.  The oil will
    fry the dough whilst baking, which is key for the characteristic
    crispy taste.  Spread the dough in the pan by stippling with
    fingers.  Allow to proof 1\fracH--2~hours (27\degreeC) or up to
    3~hours (21\degreeC).
    
  \item Bake at 220\degreeC{} for about 20~minutes.  Cool on a wire rack.

  \end{enumerate}
\end{recipe}

%\accord{}
