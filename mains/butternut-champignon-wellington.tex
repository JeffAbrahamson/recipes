\dish{Butternut Champignon Wellington}
\altdish{Wellington de butternut et champignons}
\altdish{Butternut et champignons en croûte}
\serves{6}
%\makes{}
\prep{35~minutes préparation + 60~minutes cuisson}
\source{internet}

\begin{ingredients}
  \ingr{\fracH}{}{butternut, coupé en dés}
  \ingrS{}{}{huile d'olive}
  \ingr{1}{}{oignon}
  \ingrS{1}{}{gousse d'ail}
  \ingr{500}{g}{champignon de Paris}
  \ingr{1}{\ccf}{thym}
  \ingr{1}{\ccf}{romarin}
  \ingr{50}{g}{pignon de pain}
  \ingrS{1}{\cs}{pâte de miso}
  \ingrS{100}{g}{épinards}
  \ingrS{500}{g}{\hyperref[pâte-feuilletée]{pâte feuilletée}}
\end{ingredients}


\begin{recipe}
  \begin{enumerate}

  \item Chauffer le four à 200\degreeC.

  \item Mélanger butternut avec l'huile, assaisonner, faire cuire au
    four 30--35~minutes au four jusqu'à ce qu'il soit fondant.

  \item En attendant, faire cuire les champignons avec de l’huile
    d’olive, l’oignon, l'ail et une pincée de sel.

  \item Mixer le romarin, le thym, les pignons et la pâte de miso.
    Ajouter aux champignons, puis ajouter les épinards.

  \item Dérouler la pâte.  Placer la moitié des champignons, puis la
    butternut, puis le reste des champignons.  Faire quelques petits
    trou.

  \item Enfourner 25--30~minutes.

  \end{enumerate}
\end{recipe}

