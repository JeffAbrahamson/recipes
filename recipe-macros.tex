%\usepackage[frenchb]{babel}
%\usepackage{draftcopy}
\usepackage{url}
\usepackage{makeidx}
%\usepackage[utf8]{inputenc}
\usepackage{xeCJK}
\usepackage{cjhebrew}
\usepackage{hyperref}

\makeindex

\newcommand{\headerformat}[0]{\footnotesize\hspace{4mm}}

\newenvironment{ingredients}{\bigskip\normalsize%
  \begin{tabular}%
    {l p{.8\textwidth}}}%
  {\end{tabular}}
% For recipes presented in two sizes (e.g., half batch)
\newenvironment{ingredientss}{\bigskip\normalsize%
  \begin{tabular}%
    {l | l p{.8\textwidth}}}%
  {\end{tabular}}
\newenvironment{recipe}{\bigskip\normalsize}{}

\newcommand{\dish}[1]{{\pagebreak %
    \addcontentsline{toc}{section}{#1} %
    \index{#1} %
    \begin{center}\Large #1\end{center}}}
\newcommand\subdish[1]{\index{#1} \begin{center}(\large #1)\end{center}}
\newcommand{\altdish}[1]{\index{#1}}
\newcommand{\alsodish}[1]{\centerline{\footnotesize\it (also known as)}\par%
  \centerline{\large #1} \index{#1}}
\newcommand{\aussidish}[1]{\centerline{\footnotesize\it (aussi appelé)}\par%
  \centerline{\large #1} \index{#1}}
\newcommand{\hiragana}[1]{\centerline{\footnotesize #1} \index{#1}}
%
\newcommand{\serves}[1]{\begin{center}\headerformat Serves #1 \end{center}}
\newcommand{\pour}[1]{\begin{center}\headerformat Pour #1 personnes \end{center}}
\newcommand{\pourD}[2]{\begin{center}\headerformat%
    Pour #1 personnes (#2)%
  \end{center}}
%
\newcommand{\makes}[1]{\begin{center}\headerformat Makes #1 \end{center}}
\newcommand{\fait}[1]{\begin{center}\headerformat Fait #1 \end{center}}
%
\newcommand{\prep}[1]{\bigskip{\headerformat #1 }}
\newcommand{\ingr}[3]{#1 #2& #3 \\ }
\newcommand{\ingrr}[4]{ #1  &  #2  #3  &  #4  \\ }
% An ingredient with some space after it
\newcommand{\ingrS}[3]{#1 #2& #3 \\[2mm] }
\newcommand{\ingrrS}[4]{#1 & #2 #3 & #4 \\[2mm] }
\newcommand{\ingrrT}[2]{\underline{#1} & \underline{#2} & \\[2mm] }
%
% Standard bibliographic source
\newcommand{\source}[1]{\par\headerformat Source: \textrm{\cite{#1} }%
  \normalsize }
% Bibliographic source with annotation (page number, who provided, etc.)
\newcommand{\sourceD}[2]{\par\headerformat Source: \textrm{\cite{#1}, #2 }%
  \normalsize }
% Free-form source
\newcommand{\sourcep}[1]{\par\headerformat Source: \textrm{#1}}

\newcommand{\F}[0]{~$^\circ$F}
\newcommand{\degreeC}[0]{~$^\circ$C}
\newcommand{\FC}[2]{{{#1}\F} {({#2}\degreeC})}
\newcommand{\CF}[2]{{{#1}\degreeC} {({#2}\F})}

\newcommand{\fracE}[0]{$^1\!/\!_8$ }
\newcommand{\fracEE}[0]{$^3\!/\!_8$ }
\newcommand{\fracEEE}[0]{$^5\!/\!_8$ }
\newcommand{\fracEEEE}[0]{$^7\!/\!_8$ }

\newcommand{\fracQ}[0]{$^1\!/\!_4$ }
\newcommand{\fracQQ}[0]{$^3\!/\!_4$ }

\newcommand{\fracT}[0]{$^1\!/\!_3$ }
\newcommand{\fracTT}[0]{$^2\!/\!_3$ }

\newcommand{\fracH}[0]{$^1\!/\!_2$ }

\newcommand{\cs}[0]{c. \`a soupe}
\newcommand{\ccf}[0]{c. \`a caf\'e}

\newcommand{\tsp}[0]{tsp}
\newcommand{\tbsp}[0]{Tbsp}

\newcommand\stage[1]{\bigskip\textbf{#1}}
\newcommand\notes[0]{\bigskip\textbf{Notes : }}
\newcommand\accord[1]{\textit{Le bon accord : #1.}}
