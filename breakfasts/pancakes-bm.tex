\dish{Buttermilk Pancakes}
\altdish{Pancakes, buttermilk}
\makes{9 pancakes}
\prep{10 minutes}
\source{joy}

\begin{ingredients}
  \ingr{1}{c}{all-purpose flour}
  \ingr{1}{tsp}{sugar}
  \ingr{\fracH}{tsp}{salt}
  \ingr{\fracQQ}{tsp}{double-acting baking powder}
  \ingr{\fracH}{tsp}{baking soda}
  \ingr{1}{}{egg}
  \ingr{1}{c}{buttermilk or yogurt}
  \ingr{2}{Tbsp}{melted butter}
\end{ingredients}


\begin{recipe}
  \begin{enumerate}

  \item Sift the dry ingredients into a large bowl.

  \item In a smaller bowl beat the egg until light.  Add and beat the
    buttermilk and melted butter.

  \item Make a well in the dry ingredients and Mix in the liquid with
    a few swift strokes.  (Beating it will cause the batter to be
    tough.)  If it is too thick, stir in some milk.

  \item Drop onto a hot buttered griddle.  The griddle is hot enough
    if a drop of water dances on its surface.  If the water sits and
    boils, the griddle is not hot enough.  If it disappears, it is too
    hot.  When bubbles appear on the upper surface of the pancakes,
    but before the bubbles break, turn the pancakes.  Only turn once.
    Remove from pan when second side is done, about half as long as
    required for the first side.

  \item Serve at once.  If this is not possible, store in a 200\F oven
    on a towel-covered baking sheet.  Do not stack, as off-gassing
    will make the pancakes flabby.

  \item Serve with maple syrup and extra butter.

  \end{enumerate}
\end{recipe}
