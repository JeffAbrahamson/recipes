\dish{Waffles}
\makes{9}
\prep{15}
\source{joy}

\begin{ingredients}
  \ingr{1\fracQQ}{c}{cake flour}
  \ingr{2}{tsp}{double-acting baking powder}
  \ingr{\fracH}{tsp}{salt}
  \ingr{1}{Tbsp}{sugar}
  \ingr{3}{}{eggs}
  \ingr{7}{Tbsp}{butter, melted}
  \ingr{1\fracH}{c}{milk}
\end{ingredients}


\begin{recipe}
  \begin{enumerate}

  \item Heat a waffle iron.

  \item Sift dry ingredients into a large bowl.

  \item Separate eggs.  Beat yolks until light.  Add melted butter and
    milk.  Make a well in the center of the dry ingredients and mix
    in the liquid with a few swift strokes.  The batter should have a
    pebbled look, similar to a muffin batter.  Add any mix-ins, see
    below.

  \item Beat the egg whites until stiff.  Fold into the batter until
    they are barely blended.

  \item Cover the grid until it is about \fracTT covered.  Cook about
    four minutes---steam will stop emerging from the iron when the
    waffle is done.  If the top of the iron resists opening, the
    waffle is probably not done, cook slightly longer.

  \item If the iron is, indeed, made of iron, it should be seasoned
    and then never washed (only brushed or wiped) and will need no
    additional oil or butter.

  \item At high altitudes, use about \fracQ less baking powder or soda
    than recommended.

  \end{enumerate}

Optional mix-ins:
\begin{ingredients}
  \ingr{\fracH}{c}{fresh fruit or berries}
  \ingr{\fracQ}{c}{raisins or pur\'eed dried fruit}
  \ingr{\fracQ}{c}{grated semisweet chocolate}
  \ingr{\fracQ}{c}{shredded sharp cheese}
\end{ingredients}

\end{recipe}
