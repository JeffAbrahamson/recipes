\dish{Whole Wheat Pancakes}
\altdish{Pancakes, whole wheat}
\makes{9 pancakes}
\prep{10 minutes}
\source{joy}

\begin{ingredients}
  \ingr{\fracH}{c}{all-purpose flour}
  \ingr{\fracH}{tsp}{salt}
  \ingr{\fracH}{tsp}{double-acting baking powder}
  \ingr{\fracQQ}{tsp}{baking soda}
  \ingr{1}{c}{finely milled whole wheat flour}
  \ingr{2}{Tbsp}{sugar, honey, or molasses}
  \ingr{1}{}{egg}
  \ingr{2}{c}{buttermilk or yogurt}
  \ingr{2}{Tbsp}{melted butter}
\end{ingredients}


\begin{recipe}
  \begin{enumerate}

  \item Sift into a large bowl the all-purpose flour, salt, and
    leavening.  Stir in whole wheat flour.

  \item In a smaller bowl combine and beat the sugar, egg, buttermilk,
    and melted butter.

  \item Make a well in the dry ingredients and Mix in the liquid with
    a few swift strokes.  (Beating it will cause the batter to be
    tough.)  If it is too thick, stir in some milk.

  \item Drop onto a hot buttered griddle.  The griddle is hot enough
    if a drop of water dances on its surface.  If the water sits and
    boils, the griddle is not hot enough.  If it disappears, it is too
    hot.  When bubbles appear on the upper surface of the pancakes,
    but before the bubbles break, turn the pancakes.  Only turn once.
    Remove from pan when second side is done, about half as long as
    required for the first side.

  \item Serve at once.  If this is not possible, store in a 200\F\
    oven on a towel-covered baking sheet.  Do not stack, as
    off-gassing will make the pancakes flabby.

  \item Serve with maple syrup and extra butter.

  \end{enumerate}
\end{recipe}
